\documentclass[11pt,class=report,crop=false]{standalone}
\usepackage{exo7hilisit}

\begin{document}


\entete{Hilisit}{Capacité mathématiques}

\titre{Logique, ensembles et raisonnements -- Partie 1}

\bigskip
\bigskip


\exercice{}
\enonce
Dire si les assertions suivantes sont vraies ou fausses.
 \begin{enumerate}
      \item "$6\times 7 = 42$"
      \item "$8 \times 8 = 49$"
      \item "Tout entier impair est multiple de $2$."
      \item "Tout nombre réel non nul admet un inverse."
      \item "Il existe une solution réelle de l'équation $x^2+x+1=0$."
      \item "Il existe une solution réelle de l'équation $x^2-3x+1=0$." 
\end{enumerate} 
\finenonce

\finexercice




\exercice{}
\enonce
Dire si les assertions suivantes sont vraies ou fausses.
Dans ces phrases $x$ désigne un nombre réel quelconque fixé  ($x \in \Rr$) et $n$ un entier naturel quelconque ($n \in \Nn$).

\begin{enumerate}
    \item "$(x > 0) \text{ ou } (x \le 0)$"    
    \item "$(x^2 \ge 0) \text{ ou } (x^2 \le 0)$"
    \item "($n$ est divisible par $2$) ou ($n$ est divisible par $3$)"
    \item "non($x^4 < 0$)"
    \item "$(x > 3) \text{ ou } \text{non}(x \ge 4)$"  
    \item "($n$ est impair) et ($n$ est divisible par $2$)"
    \item "($n$ est pair) ou (non($n$ est divisible par $2$))"
    \item "$(x>0)  \text{ ou } (x<0)  \text{ ou } (x=0)$"
    \item "$(x>0)  \text{ et } (\text{non}(x>0))$".
\end{enumerate}
 
\finenonce

\finexercice



\exercice{}
\enonce
Dire si les assertions suivantes sont vraies ou fausses.
\begin{enumerate}
    \item $\forall n \in \Nn  \quad n \text{ est un nombre premier }$
    \item $\exists n \in \Nn  \quad n \text{ est un nombre premier }$
    \item $\forall x \in \Rr  \quad x^2+1 \ge 1$      
    \item $\exists x \in \Rr  \quad x^2+1 \ge 1$  
    \item $\forall x \in \Rr^*_+  \quad x > \frac 1x$         
    \item $\exists x \in \Rr^*_+  \quad x > \frac 1x$
    \item $\exists x \in \Rr  \quad x^2+x-1 = 0$         
    \item $\exists x \in \Zz  \quad x^2+x-1 = 0$                 
\end{enumerate} 
\finenonce


\finexercice



\exercice{}
\enonce
Remplacer les pointillés des propositions suivantes par le symbole le plus adapté parmi $\implies$, $\impliedby$ ou $\iff$.

Dans ces phrases $x$ et $y$ désignent des nombres réels et $n$ un entier naturel.

\begin{enumerate}
    \item $x > 0  \qquad \ldots\ldots \qquad x^2 > 0$
    \item $ -x < 0 \qquad \ldots\ldots \qquad 3x > 1$    
    \item $x^2 = 4  \qquad \ldots\ldots \qquad (x = 2)  \text{ ou } (x = -2)$      
    \item $x \neq y  \qquad \ldots\ldots \qquad x^2 \neq y^2$
    \item $xy = 0  \qquad \ldots\ldots \qquad x = 0 \text{ ou } y = 0$    
    \item $xy = 0  \qquad \ldots\ldots \qquad x = 0 \text{ et } y = 0$
    \item $xy \neq 0  \qquad \ldots\ldots \qquad x \neq 0 \text{ ou } y \neq 0$  
    \item $n \ge 3$ et $n$ impair  \qquad \ldots\ldots \qquad $n \ge 3$ et $n$ un nombre premier 
    \item $n \ge 3$ et $n$ pair  \qquad \ldots\ldots \qquad $n \ge 3$ et $n$ n'est pas un nombre premier      
\end{enumerate} 
\finenonce

\finexercice




\exercice{}
\enonce
Écrire la contraposée de chacune des propositions suivantes. 
Dans ces phrases, $x$ désigne un réel et $n$ un entier naturel quelconque.
(On ne demande pas de dire si les phrases sont vraies ou fausses.)
\begin{enumerate}
    \item Il pleut $\implies$ Je prends mon parapluie
    \item $x^2 \neq 0 \implies x \neq 0$ 
    \item $7x-1 > 20 \implies x > 3$ 
    \item $n^2$ est pair $\implies$ $n$ est pair 
    \item Si un triangle est rectangle alors le carré de l'hypoténuse est égal à la somme des carrés des côtés opposés.
\end{enumerate} 
\finenonce

\finexercice





\end{document}
