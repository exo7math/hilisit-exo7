\documentclass[11pt,class=report,crop=false]{standalone}
\usepackage{exo7hilisit}

\newcommand{\pgcd}{\mathop{\mathrm{pgcd}}\nolimits} 
\newcommand{\ppcm}{\mathop{\mathrm{ppcm}}\nolimits}

\begin{document}
    


\entete{Hilisit}{Capacité mathématiques}

\titre{Arithmétique -- Partie 1 : pgcd}

\bigskip
\bigskip


%%%%%%%%%%%%%%%%%%%%%%%%%%%%%%%%%%%%%%%%%%%%%%%%%%%%%%%%%%%%
%\section{pgcd}

\exercice{}
\enonce
\sauteligne
 \begin{enumerate}
  \item Chercher le plus petit entier positif divisible par $11$ et dont le reste de la division par $13$ est $1$.
  
  \item Chercher le plus petit entier positif dont le reste de la division par $8$ est $5$ et le reste de la division par $9$ est $6$.
\end{enumerate} 
\finenonce


\finexercice


\exercice{}
\enonce
\sauteligne
\begin{enumerate}
    \item Soit $n=p^2$ le carré d'un entier. Quel peut être le reste de la division de $n$ par $4$ selon que $p$ est pair ou impair ?
    \item Montrer que si $n$ est un entier naturel somme de deux carrés d'entiers 
    alors le reste de la division de $n$ par $4$ n'est jamais égal à $3$.
\end{enumerate}


\finenonce 


\finexercice



\exercice{}
\enonce
Déterminer $\pgcd(254, 26)$, $\pgcd(654, 115)$ à l'aide de l'algorithme d'Euclide.
\finenonce


\finexercice



\exercice{}
\enonce
Déterminer $\ppcm(255, 204)$.
\finenonce


\finexercice


\end{document}

