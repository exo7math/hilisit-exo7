\documentclass[11pt,class=report,crop=false]{standalone}
\usepackage{exo7hilisit}

\usepackage{cancel}

\begin{document}

%%%%%%%%%%%%%%%%%%%%%%%%%%%%%%%%%%%%%%%%%%%%%%%%%%%%%%%%%%%%%%%%%%%%%%
%%%%%%%%%%%%%%%%%%%%%%%%%%%%%%%%%%%%%%%%%%%%%%%%%%%%%%%%%%%%%%%%%%%%%%


\entete{Hilisit}{Capacité mathématiques}

\titre{Arithmétique -- Partie 4 : Congruences} 

\encadre{
	\emph{Savoir.}
	\begin{itemize}[label=$\square$]
		\item Comprendre la définition de la congruence.
        \item Connaître le petit théorème de Fermat.
	\end{itemize}
	\emph{Savoir-faire.}
	\begin{itemize}[label=$\square$]
		\item Savoir faire des calculs modulo $n$. 
	\end{itemize}
}



\bigskip


%-----------------------------------------
\subsection*{Congruences}


\textbf{Définition.}
Soient $a$ et $b$ deux entiers et un entier naturel $n\geq2$. 
On dit que \boldmath \textbf{$a$ est congru à $b$ modulo $n$ }\unboldmath si $n$ divise la différence $(b-a)$. On note alors :
\mybox{$a \equiv b  \;[n]$}

\emph{Remarques.}
\begin{itemize}
        \item $a \equiv b \;[n]$ revient à dire que les restes de $a$ et de $b$ dans la division euclidienne par $n$ sont les mêmes. Cela veut aussi dire que $a$ et $b$ ne diffèrent que d'un multiple de $n$, ce qui s'écrit $b = a + kn, \; k \in \Zz$.
        
        C'est ainsi que $13 \equiv 8 \equiv 3 \equiv -2 \; [5]$ puisque tous ces nombres ne diffèrent entre eux que de multiples de $5$.
                
        \item On voit parfois dans les livres la notation $a\equiv b \pmod n$.
        
        \item $a \equiv 0 \; [n]$ signifie que $n|a$.
\end{itemize}

\emph{Exemples.}

\begin{itemize}
        \item $ 65 \equiv 2 \; [7]$. En effet $7$ divise $65-2=63$ (ou encore $65=7\times9+\boldsymbol{2}$).
        
        \item $ 13\,145 \equiv 165 \equiv 5 \; [10]$. En fait un nombre entier est congru à un autre modulo $10$ s'ils se terminent par le même chiffre.
        
        \item $n\equiv 0 \; [2]$ signifie que $n$ est pair, et $n\equiv 1 \; [2]$ que $n$ est impair.\\
        Il n'y a donc que $2$ possibilités modulo $2$ : être congru à $0$ ou à $1$. De même il n'y a que trois possibilités modulo $3$ : être congru à $0,1$ ou $2$ (ou encore $0,1$ et $-1$ car $-1\equiv 2 \;[3]$ !).\\ De manière générale, il y a $n$ possibilités modulo $n$.
\end{itemize}


%-----------------------------------------
\subsection*{Calculs}


Les congruences sont bien adaptées aux additions, soustractions et aux multiplications : autrement dit, on peut y faire de l'arithmétique.

\medskip

\textbf{Les règles de calcul.}
Si $a \equiv b \; [n]$ et $c \equiv d \; [n]$, alors :
\mybox{$a+c \equiv b+d \; [n]$ \quad (addition)}
et aussi    
$$a-c \equiv b-d  \; [n] \quad  \text{(soustraction)}$$
enfin
\mybox{$ac \equiv bd \; [n]$ \quad (multiplication)}

\emph{Attention.}Il n'est pas question de parler d'une éventuelle opération de division dans le monde du \textit{modulo} ! En effet, ce monde se préoccupe exclusivement des nombres \textbf{entiers} ! On s'échapperait de ce monde merveilleux si on se hasardait à y tenter de la division... Par exemple il serait \textbf{extrêmement FAUX} de dire que :
$$  2 \equiv 12 \; [10] \quad \bcancel{\implies \quad \frac{2}{12} \equiv 1 \; [10]  } $$  
Aussi on ne peut pas simplifier, par exemple ci-dessous diviser par $2$ n'a pas de sens :
$$  6 \equiv 2 \; [4]  \quad \bcancel{\implies \quad 3 \equiv 1 \; [4]  } $$


\emph{Démonstration des règles de calcul.}
\begin{itemize}
    \item $n|(b-a)$ et $n|(d-c)$, donc $n$ divise l'addition $(b-a)+(d-c)=(b+d)-(a+c)$ (d'où la règle d'addition) et la soustraction $(b-a)-(d-c)=(b-d)-(a-c)$ (d'où la règle de soustraction).
    
    \item Pour la multiplication, $n|(b-a)$ donc $n|d(b-a)=db-da$ d'une part ;
     et $n|(d-c)$ donc $n|a(d-c)=ad-ac$ d'autre part. Par addition, $n|db-da+ad-ac=db-ac$ d'où la règle de multiplication.
\end{itemize} 


\bigskip

\textbf{Corollaire.} 
Si $a \equiv b \; [n]$, alors pour tout entier $l$ : $la\equiv lb \; [n]$
et pour tout entier $k$ positif, on a : 
\mybox{$a^k \equiv b^k \; [n]$}



\emph{Exemples.}

\begin{itemize}
    \item Commençons par déterminer si le nombre $4^{48}-1$ est ou non un multiple de $5$.\\
    Il s'agit de voir si $4^{48}-1 \equiv 0 \; [5]$ est vraie.\\
    Puisque $4^2=16=3\times5+1$, on obtient $4^2\equiv 1\;[5]$. On passe cette égalité à la puissance $24$ (d'après le corollaire) puis on retranchera $1$ (d'après les règles de soustraction) :
    $$ 4^2 \equiv 1 \;[5] \implies (4^2)^{24} \equiv 1^{24} \; [5] \iff 4^{48} \equiv 1 \; [5] \iff 4^{48}-1 \equiv 0 \;[5] $$
    Ainsi le nombre $4^{48}-1$ est bien un multiple de $5$. Parviens-tu, en utilisant les congruences modulo $2$, à déterminer s’il se termine par $0$ ou par $5$ ?
    
    \item Cherchons à présent quel est le chiffre des unités du nombre $3^{240}+7^{240}$. Il s'agit de déterminer la congruence de ce nombre modulo $10$.\\
    Tout d'abord, $3^2=9$ est congru à $-1$ modulo $10$. 
    
    Ce sera notre point de départ :
    $$ 3^2 \equiv -1 \; [10] \implies (3^2)^{120}\equiv (-1)^{120} \; [10] \iff 3^{240} \equiv 1 \; [10] $$
    On rappelle que calculer $(-1)^k$ est facile : c'est $+1$ si $k$ est pair et $-1$ si $k$ est impair.
    Passons à la puissance de $7$ : puisque $7^2=49$ est congru à $-1$ modulo $10$, on obtient de même :
    $$ 7^2 \equiv -1 \; [10] \implies (7^2)^{120}\equiv (-1)^{120} \; [10] \iff 7^{240} \equiv 1 \; [10]$$
    Ainsi par addition, $3^{240}+7^{240} \equiv 2 \; [10]$ ce qui signifie que son chiffre des unités est $2$.
\end{itemize}    

\bigskip

\emph{Exercice.}
\begin{enumerate}
    \item Montrer que pour tout entier naturel $n$, le nombre $4^{3n}-4^n$ est un multiple de $5$.
    
    \item Montrer que "$13 | (5^{2n}+3^{3n}) \iff n$ est impair".
\end{enumerate}    
    



%-----------------------------------------
\subsection*{Petit théorème de Fermat}


Le calcul des congruences est particulièrement intéressant lorsque le modulo choisi est lui-même un nombre premier. C'est notamment ce qu'illustre le petit théorème de Fermat :

\textbf{Petit théorème de Fermat ($1\,640$).}
Si $p$ est un nombre premier, alors pour tout entier $x$ on a : 
\mybox{$x^p \equiv x \;[p]$}

En particulier, \textbf{si $x$ n'est pas un multiple de $p$}, alors :
\mybox{$x^{p-1} \equiv 1 \; [p]$}

\medskip

\emph{Exemple.}
Calculons $5^{2\,022}$ modulo $13$.
\begin{itemize}
    \item D'après le petit théorème de Fermat, $13$ étant premier et $5$ n'étant pas un multiple de $13$, on sait que $5^{12} \equiv 1 \;[13]$.
    
    \item Par ailleurs la division euclidienne de $2\,022$ par $12$ est  = $2\,022 = 12\times 168+6$. Donc en passant à la puissance $168$ on obtient $5^{12\times 168} = (5^{12})^{168} \equiv 1 \;[13]$.
    
    \item Or $5^2\equiv25\equiv-1 \;[13]$,  ainsi $5^6 = (5^2)^3  \equiv (-1)^3  \equiv -1 \;[13]$.\\
    Par multiplication, on obtient : \quad $5^{2\,022}  \equiv 5^{12\times 168} \times 5^6  \equiv 1 \times (-1 ) \equiv -1 \equiv 12 \; [13]$.
\end{itemize}

\medskip

\emph{Exercice.} Calculer $4^{253}$ modulo $11$ et $25^{71}$ modulo $7$.



\end{document}
