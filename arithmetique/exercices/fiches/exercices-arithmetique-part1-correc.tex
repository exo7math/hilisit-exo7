\documentclass[11pt,class=report,crop=false]{standalone}
\usepackage{exo7hilisit}

\newcommand{\pgcd}{\mathop{\mathrm{pgcd}}\nolimits} 
\newcommand{\ppcm}{\mathop{\mathrm{ppcm}}\nolimits}

\begin{document}
    


\entete{Hilisit}{Capacité mathématiques}

\titre{Arithmétique -- Partie 1 : pgcd}

\bigskip
\bigskip


%%%%%%%%%%%%%%%%%%%%%%%%%%%%%%%%%%%%%%%%%%%%%%%%%%%%%%%%%%%%
%\section{pgcd}

\exercice{}
\enonce
\sauteligne
 \begin{enumerate}
  \item Chercher le plus petit entier positif divisible par $11$ et dont le reste de la division par $13$ est $1$.
  
  \item Chercher le plus petit entier positif dont le reste de la division par $8$ est $5$ et le reste de la division par $9$ est $6$.
\end{enumerate} 
\finenonce

\indication
Commencer par écrire tous les multiples de $11$ et effectuer ensuite la division euclidienne par $13$.
\finindication

\correction
\sauteligne
\begin{enumerate}
  \item 
Les entiers divisibles par $11$ sont les multiples de $11$ : $0$, $11$, $22$, \ldots{} Ils sont de la forme $11k$ pour un certain entier $k$.

$$\begin{array}{c|c|c}
k & 11 k & \text{reste par $13$} \\ \hline
1  &  11  &  11 \\
2  &  22  &  9 \\
3  &  33  &  7 \\
4  &  44  &  5 \\
5  &  55  &  3 \\
6  &  66  &  1 \\
\end{array}$$

On note sur la dernière colonne que le reste de $11k$ divisé par $13$ diminue ici de deux en deux et pour $k=6$ on obtiendra le reste $1$. Ainsi le nombre cherché est $n=66$ : c'est un multiple de $11$ et le reste de la division par $13$ est bien $1$ car $66 = 13 \times 5 +1$.

  
  \item Les entiers dont le reste de la division par $8$ est $5$ sont de la forme $8k+5$ pour un certain entier $k$. Reprenons la même méthode, on calcule tous les entiers de la forme $8k+5$ et le reste de division par $9$ :
$$\begin{array}{c|c|c}
k & 8 k + 5 & \text{reste par $9$} \\ \hline
0  &  5  &  5 \\
1  &  13  &  4 \\
2  &  21  &  3 \\
3  &  29  &  2 \\
4  &  37  &  1 \\
5  &  45  &  0 \\
6  &  53  &  8 \\
7  &  61  &  7 \\
8  &  69  &  6 \\
\end{array}$$  
 On note sur la dernière colonne que le reste "diminue de $1$" à chaque ligne et pour  $k=8$ on obtiendra le reste $6$. Ainsi le nombre cherché est $n=8\times8+5=69$ qui s'écrit aussi $69=6\times9+6$. 
\end{enumerate}
\fincorrection
\finexercice


\exercice{}
\enonce
\sauteligne
\begin{enumerate}
    \item Soit $n=p^2$ le carré d'un entier. Quel peut être le reste de la division de $n$ par $4$ selon que $p$ est pair ou impair ?
    \item Montrer que si $n$ est un entier naturel somme de deux carrés d'entiers 
    alors le reste de la division de $n$ par $4$ n'est jamais égal à $3$.
\end{enumerate}


\finenonce 

\indication
Si $p$ est pair, alors $p=2k$ donc $p^2 = \ldots${}
Si $p$ est impair, alors $p=2k+1$\ldots
\finindication

\correction
\sauteligne
\begin{enumerate}
    \item Soit $n=p^2$. 
    \begin{itemize}
      \item Si $p$ est pair, alors $p=2k$ (pour un certain entier $k$) donc $n=p^2 = (2k)^2 = 4k^2$ est un multiple de $4$. Dans ce cas le reste de la division de $n$ par $4$ est $0$.
    
      \item Si $p$ est impair, alors $p=2k+1$ donc $n=p^2 = (2k+1)^2 = 4k^2+4k+1 = 4(k^2+k) + 1$, c'est l'écriture de la division euclidienne de $n$ par $4$. Donc le reste de la division de $n$ par $4$ est $1$.
    
      \item Conclusion : pour $n=p^2$ alors le reste de la division de $n$ par $4$ est soit $0$, soit $1$ (mais ne peut pas être $2$, ni $3$).
    \end{itemize}
    
    \item Soit $n=p^2+q^2$.
    On discute selon que $p$ et $q$ sont pairs ou impairs. Il y a donc $4$ cas possibles.
    \begin{itemize}
         \item Si $p$ est pair et $q$ est pair. Alors par la question précédente le reste de la division de $p^2$ par $4$ est $0$, de même que celui de la division de $q^2$ par $4$. Ainsi le reste de la division de $n=p^2+q^2$ est $0+0$, il vaut donc $0$.
         
         \item Si $p$ est pair et $q$ est impair, alors le reste de la division de $n=p^2+q^2$ est $0+1$, il vaut donc $1$.
           
         \item Si $p$ est impair et $q$ est pair, alors le reste de la division de $n=p^2+q^2$ est $1+0$, il vaut donc $1$.       
         
         \item Si $p$ est impair et $q$ est impair, alors le reste de la division de $n=p^2+q^2$ est $1+1$, il vaut donc $2$.  
   \end{itemize}   

    Dans tous les cas le reste de $n$ divisé par $4$ ne peut pas être $3$.
    
\end{enumerate}
\fincorrection
\finexercice



\exercice{}
\enonce
Déterminer $\pgcd(254, 26)$, $\pgcd(654, 115)$ à l'aide de l'algorithme d'Euclide.
\finenonce

\indication
Calculer une succession de divisions euclidiennes.
\finindication

\correction
\sauteligne
\begin{enumerate}
    \item   
    Calculons $\pgcd(254,26)$.
    $$
    \begin{array}{rclclcl}
    254 & = & 26 & \times & 9 & + & 20 \\
    26 & = & 20  & \times & 1 & + & 6 \\
    20  & = & 6  & \times & 3 & + & \boxed{2} \\
    6  & = & 2  & \times & 3 & + & 0 \\
    \end{array}
    $$
    Ainsi $\pgcd(254,26) = 2$.
    
    
    \item 
    Calculons $\pgcd(654,115)$.
$$
\begin{array}{rclclcl}
654 & = & 115 & \times & 5 & + & 79 \\
115 & = & 79  & \times & 1 & + & 36 \\
79 & = & 36  & \times & 2 & + & 7 \\
36 & = & 7  & \times & 5 & + & \boxed{1}  \\
7  & = & 1  & \times & 7 & + & 0 \\
\end{array}
$$
Ainsi $\pgcd(654,115) = 1$.    
    

\end{enumerate}
\fincorrection
\finexercice



\exercice{}
\enonce
Déterminer $\ppcm(255, 204)$.
\finenonce

\indication
Utiliser le lien entre $\pgcd$ et $\ppcm$...
\finindication

\correction
\sauteligne
On va utiliser la relation
$$ 255 \times 204 = \pgcd(255,204) \times \ppcm(255,204) $$ 
    Calculons donc $\pgcd(255,204)$.
    $$
    \begin{array}{rclclcl}
    255 & = & 204 & \times & 1 & + & \boxed{51} \\
    204 & = & 51  & \times & 4 & + & 0 \\
    \end{array}
    $$
    Ainsi $\pgcd(255,204) = 51$. On en déduit donc :
    $$ \ppcm(255,204) = \frac{255 \times 204}{\pgcd(255,204)} = \frac{255 \times 204}{51} = \boxed{ \; 1020 \;} $$
\fincorrection
\finexercice



\end{document}
