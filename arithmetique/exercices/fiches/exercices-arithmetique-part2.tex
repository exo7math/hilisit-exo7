\documentclass[11pt,class=report,crop=false]{standalone}
\usepackage{exo7hilisit}

\newcommand{\pgcd}{\mathop{\mathrm{pgcd}}\nolimits} 
\newcommand{\ppcm}{\mathop{\mathrm{ppcm}}\nolimits}

\begin{document}
    


\entete{Hilisit}{Capacité mathématiques}

\titre{Arithmétique -- Partie 2 : Théorème de Bézout}

\bigskip
\bigskip



%%%%%%%%%%%%%%%%%%%%%%%%%%%%%%%%%%%%%%%%%%%%%%%%%%%%%%%%%%%%
%\section{Théorème de Bézout}


\exercice{}
\enonce
Soit $a=84$ et $b=75$. Calculer $d=\pgcd(a,b)$ à l'aide de l'algorithme d'Euclide, puis déterminer des coefficients de Bézout $u,v \in \Zz$ tels que $au+bv=d$.

Même exercice avec $a=624$ et $b=108$.
\finenonce

\finexercice




\exercice{}
\enonce
Nous allons montrer que \og{}Deux entiers consécutifs sont toujours premiers entre eux.\fg{}
\begin{enumerate}
    \item \emph{Première méthode.} On considère deux entiers consécutifs notés $n$ et $n+1$. Montrer que si $d$ divise $n$ et $n+1$ alors nécessairement $d=1$. 
    
    \item \emph{Seconde méthode.} Soit $a=n$ et $b=n+1$. Trouver $u,v \in \Zz$ (très simples) tels que $au+bv=1$. Conclure.
\end{enumerate} 
\finenonce

\finexercice



\exercice{}
\enonce
Soit $\alpha \in \Zz$ un entier fixé que l'on cherchera à déterminer par la suite. Pour $k \in \Zz$, on pose : $N_1(k) = 7k+11$ et $N_2(k) = 2k + \alpha$.
\begin{enumerate}
    \item Déterminer deux entiers $u,v$ tels que le nombre $u N_1(k) + v N_2(k)$ ne dépende pas de l'entier $k$.
    
    \item En déduire une valeur de $\alpha$ pour obtenir $\pgcd(N_1(k) , N_2(k)) = 1$ pour tout entier $k$.
    
    \item Application : en déduire $\pgcd(95,27)$ d'une part, et $\ppcm(361,103)$ d'autre part.
\end{enumerate}
\finenonce


\finexercice

\end{document}

