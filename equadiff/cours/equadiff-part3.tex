\documentclass[11pt,class=report,crop=false]{standalone}
\usepackage{exo7hilisit}

\begin{document}

%%%%%%%%%%%%%%%%%%%%%%%%%%%%%%%%%%%%%%%%%%%%%%%%%%%%%%%%%%%%%%%%%%%%%%
%%%%%%%%%%%%%%%%%%%%%%%%%%%%%%%%%%%%%%%%%%%%%%%%%%%%%%%%%%%%%%%%%%%%%%

\entete{Hilisit}{Capacité mathématiques}

\titre{\'Equations différentielles -- Partie 3 : $y'=ay$} 

\encadre{
	\emph{Savoir.}
	\begin{itemize}[label=$\square$]
		\item Connaître la formule de la solution d'une équation différentielle $y'=ay$.
		\item Comprendre ce qu'est une condition initiale d'une équation différentielle.
		\item Comprendre qu'il y a unicité d'une solution lorsqu'on impose une condition initiale.
%        \item Savoir interpréter l'unicité en terme des courbes solutions qui ne s'intersectent pas.		
	\end{itemize}
	\emph{Savoir-faire.}
	\begin{itemize}[label=$\square$]
		\item Savoir résoudre une équation différentielle $y'=ay$.
		\item Savoir trouver la solution vérifiant une condition initiale donnée.
	\end{itemize}
}


\setcounter{equation}{0}


\bigskip


%%%%%%%%%%%%%%%%%%%%%%%%%%%%%%%%%%%%%%%%%%%%%%%%%%%
\subsection*{Un exemple}

Une équation différentielle a en général une infinité de fonctions solutions.

Considérons par exemple l'équation différentielle :
\begin{equation}
y'(x) = y(x)
\label{eq:eqdiff1}
\end{equation}

Alors les solutions de (\ref{eq:eqdiff1}) sont les fonctions :
$$y(x) = Ce^x \qquad \text{ où } C \in \Rr.$$ 
Ainsi, chaque valeur de la constante $C$ fournit une fonction solution que l'on note $y_C$ :
par exemple pour $C=1$, $y_1(x) = e^x$ est solution, pour $C=-2$, 
$y_{-2}(x) = -2e^x$ est solution, pour $C=0$, $y_0(x)=0$ est solution\ldots


Pour n'avoir qu'une seule solution, on peut imposer une condition initiale
\begin{equation}
\left\{\begin{array}{lr}
y'(x) = y(x) &\qquad \text{équation différentielle}\\
y(0) = 3    &\qquad \text{condition initiale}
\end{array}\right.
\label{eq:eqdiff2}
\end{equation}

Les solutions de l'équation différentielle sont de la forme $y(x)=Ce^x$,
mais on veut $y(0)=3$. 
Comme $y(0)=Ce^0=C$, on doit avoir $C=3$. 
Ainsi l'unique solution du problème (\ref{eq:eqdiff2}) est la fonction 
$y(x) = 3e^x$.

\medskip

\emph{Exercice.}
Considérons l'équation différentielle avec condition initiale :
\begin{equation}
\left\{\begin{array}{l}
y'(x) = y(x)\\
y(1) = 2    
\end{array}\right.
\label{eq:eqdiff3}
\end{equation}
Trouver l'unique solution de ce problème. (Attention ce n'est pas $y(x) = 2e^x$ !)

%%%%%%%%%%%%%%%%%%%%%%%%%%%%%%%%%%%%%%%%%%%%%%%%%%%
\subsection*{$\boldsymbol{y'=ay}$}

Considérons l'équation différentielle $y'=ay$ où $a\in\Rr$ est une constante fixée.
Cette équation s'appelle une \emph{équation différentielle homogène linéaire d'ordre $1$ à coefficients constants}.


\mybox{\textbf{Théorème.} Les solutions de l'équation différentielle $y'=ay$ sont les fonctions définies par $f(x) = Ce^{ax}$ où $C\in\Rr$ est une constante.}

\emph{Exemples.}
\begin{itemize}
  \item $y'=6y$ : ici $a=6$ donc les solutions sont les fonctions $x \mapsto Ce^{6x}$ quelle que soit la constante $C$. Il y a donc une infinité de solutions par exemple $x \mapsto 8e^{6x}$ (pour $C=8$), $x \mapsto \pi e^{6x}$ (pour $C=\pi$), $x \mapsto 0$ (pour $C=0$)\ldots

  \item $2y'+4y=0$ : cette équation s'écrit aussi $y'=-2y$. Ici $a=-2$ (attention au signe !), les solutions sont les fonctions $Ce^{-2x}$ où $C$ est une constante réelle.
\end{itemize}


%%%%%%%%%%%%%%%%%%%%%%%%%%%%%%%%%%%%%%%%%%%%%%%%%%%
\subsection*{Condition initiale}

\subsubsection*{Définition}
Pour une équation différentielle $y'=ay$, une \textbf{condition initiale} c'est le fait d'imposer une égalité du type :
$$y(x_0)=y_0$$
où $x_0\in\Rr$ et $y_0\in\Rr$ sont des constantes.


\mybox{\textbf{Théorème d'unicité.} Une équation différentielle $y'=a y$ avec condition initiale admet une unique solution.}


\emph{Exemple.}
$$y'(x)=2y(x) \quad\text{ et }\quad y(0)=4$$

Les solutions de l'équation différentielle $y'=2y$ sont les fonctions définies par 
$f(x) = Ce^{2x}$ où $C\in \Rr$.
Mais nous voulons en plus que la fonction solution vérifie la condition initiale $f(0)=4$. Cela entraîne $Ce^{0} = 4$, donc $C=4$. Ainsi la seule solution au problème est la fonction $f(x) = 4 e^{2x}$.
Pour se rassurer, c'est une bonne idée de vérifier que $f'=2f$ et $f(0)=4$.

 

%%%%%%%%%%%%%%%%%%%%%%%%%%%%%%%%%%%%%%%%%%%%%%%%%%%
\subsection*{Courbes solutions}

Une \textbf{courbe solution} d'une équation différentielle $(E)$
est le graphe d'une solution de $(E)$.


Pour l'équation différentielle 
$$y'(x) = y(x)$$
on sait que les solutions sont les $y(x) = Ce^x$, où $k\in \Rr$ est une constante. Ci-dessous sont tracés quelques graphes de ces solutions.

\begin{center}
\begin{tikzpicture}[scale=0.8]

  \draw[->,>=latex,thick,gray] (-6.5,0) -- (2.4,0) node[below left,black] {$x$};
  \draw[->,>=latex,thick,gray] (0,-5) -- (0,5) node[left,black] {$y$};

\begin{scope}[xscale=1]
\foreach \k in {-3,-2.5,...,3} {
  \draw[thick, color=myred,domain=-6:2, smooth,samples=50] plot (\x,{\k*exp(+0.33*\x)});
}
\end{scope}

%\node[blue] at (-3,3) {Cas \  $a>0$};

\draw[blue] (2.3,6)--(2.5,6)--(2.5,0.2)--(2.3,0.2);
\draw[blue] (2.3,-6)--(2.5,-6)--(2.5,-0.2)--(2.3,-0.2);
\node[blue, right] at (3,3) {$C>0$};
\node[blue, right] at (3,0) {$C=0$};
\node[blue, right] at (3,-3) {$C<0$};

\end{tikzpicture}
\end{center}

Pour une équation $y'=ay$ le théorème d'unicité se reformule ainsi :
\mybox{
\og Par chaque point $(x_0,y_0) \in \Rr^2$
passe une et une seule courbe solution. \fg}

En particulier :
\mybox{
\og 
Deux courbes solutions ne s'intersectent pas.
\fg}


\emph{Exemple.}
Les solutions de l'équation différentielle
$3y'=-y$ sont les
$$f(x) = Ce^{-\frac13x} \quad C \in\Rr.$$

Pour chaque point $(x_0,y_0) \in \Rr^2$, il existe une unique solution
$y$ telle que $y(x_0)=y_0$. Le graphe de cette solution
est la courbe solution passant par $(x_0,y_0)$.

\begin{center}
\begin{tikzpicture}

  \draw[->,>=latex,thick,gray] (-5.5,0) -- (6.5,0) node[below,black] {$x$};
  \draw[->,>=latex,thick,gray] (0,-4.5) -- (0,5) node[left,black] {$y$};
\begin{scope}
    \clip (-5,-4) rectangle (5,4);
\begin{scope}%[xscale=1.5]

\foreach \k in {-4,-3.5,...,4} {
  \draw[thick, color=myred,domain=-5:5, smooth,samples=10] plot (\x,{\k*exp(-0.33*\x)});
}
\foreach \k in {4,4.5,...,20} {
  \draw[thick, color=myred,domain=-1:5, smooth,samples=10] plot (\x,{\k*exp(-0.33*\x)});
}
\foreach \k in {-4,-4.5,...,-20} {
  \draw[thick, color=myred,domain=-1:5, smooth,samples=10] plot (\x,{\k*exp(-0.33*\x)});
}

\def\k{3}
\draw[ultra thick, color=myred,domain=-2:5, smooth,samples=10] plot (\x,{\k*exp(-0.33*\x)});
\end{scope}
\end{scope}

\fill[blue] (1,2.15)  circle (2pt) node [below] {$(x_0,y_0)$}; 

\draw (-5,-4) rectangle (5,4);
\end{tikzpicture}
\end{center}


\end{document}
