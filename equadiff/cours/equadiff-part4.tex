\documentclass[11pt,class=report,crop=false]{standalone}
\usepackage{exo7hilisit}

\begin{document}

%%%%%%%%%%%%%%%%%%%%%%%%%%%%%%%%%%%%%%%%%%%%%%%%%%%%%%%%%%%%%%%%%%%%%%
%%%%%%%%%%%%%%%%%%%%%%%%%%%%%%%%%%%%%%%%%%%%%%%%%%%%%%%%%%%%%%%%%%%%%%


\entete{Hilisit}{Capacité mathématiques}

\titre{\'Equations différentielles -- Partie 4 : $\boldsymbol{y'=ay+b}$ et $\boldsymbol{y'=ay+f}$} 

\encadre{
	\emph{Savoir.}
	\begin{itemize}[label=$\square$]
		\item Connaître le vocabulaire : second membre, équation homogène, solution particulière.
		\item Comprendre la structure d'une solution générale : "solutions de l'équation homogène + solution particulière".
	\end{itemize}
	\emph{Savoir-faire.}
	\begin{itemize}[label=$\square$]
		\item Savoir trouver une solution particulière à l'aide d'indications.
		\item Savoir trouver toutes les solutions à partir d'une solution particulière et des solutions de l'équation homogène.
	\end{itemize}
}


\setcounter{equation}{0}

Nous allons maintenant résoudre des équations différentielles un peu plus compliquées du type $y'=ay + b$ (où $a$ et $b$ sont des constantes) et plus généralement du type $y'=ay + f$ où $a$ est une constante mais où $f$ est une fonction.

%%%%%%%%%%%%%%%%%%%%%%%%%%%%%%%%%%%%%%%%%%%%%%%%%%%
\subsection*{Vocabulaire}

Considérons l'équation différentielle 
\begin{equation}
y'=ay + b
\tag{$E$}
\end{equation}
où $a,b \in \Rr$,  ou bien plus généralement
\begin{equation}
y'=ay + f
\tag{$E$}
\end{equation}
où $a \in \Rr$ et $x \mapsto f(x)$ est une fonction.


\emph{Vocabulaire.}
\begin{itemize}
  \item $b$ ou $f(x)$ s'appelle le \textbf{second membre} de l'équation $(E)$ (cette terminologie se justifie car l'équation peut aussi s'écrire $y'-ay=b$ ou $y'-ay=f$ en mettant d'un côté tous les termes en $y$).

  \item Une \textbf{solution particulière} c'est n'importe quelle fonction $y_p(x)$ solution de l'équation originelle $(E)$.

  \item L'\textbf{équation homogène} associée à $(E)$ est 
\begin{equation}
y'=ay
\tag{$E_h$}
\end{equation}
que l'on peut aussi écrire $y'-ay=0$. C'est l'équation de départ sans son second membre. On note $y_h(x)$ les solutions de l'équation homogène.
\end{itemize}


\emph{Exemple 1.}
Soit l'équation $(E)$ : $y' = 2y + 7$.
\begin{itemize}
  \item L'équation homogène est $(E_h)$ : $y'=2y$.
  \item Le second membre est $b=7$.
\end{itemize}

\emph{Exemple 2.}
Soit l'équation $(E)$ : $3y' +7y = 2\cos(x)$.
\begin{itemize}
  \item L'équation homogène est $(E_h)$ : $3y'+7y=0$, que l'on peut aussi écrire $y' = -\frac73 y$.
  \item Le second membre est $f(x)=2\cos(x)$.
\end{itemize}



%%%%%%%%%%%%%%%%%%%%%%%%%%%%%%%%%%%%%%%%%%%%%%%%%%%
\subsection*{Structure des solutions}

Reprenons une équation différentielle avec second membre
\begin{equation}
y'=ay + b \qquad \text{ ou } \qquad y'=ay +f
\tag{$E$}
\end{equation}


On sait résoudre l'équation homogène associée $(E_h)$ : $y'=ay$, les solutions sont les fonctions $y_h(x) = Ce^{ax}$ où $C\in\Rr$.

Imaginons que l'on connaisse en plus une solution particulière $y_p(x)$ de l'équation originale $(E)$.


\mybox{\textbf{Solutions d'une équation différentielle avec second membre.} \\
Les solutions de $(E)$ sont les fonctions 
$y(x) = y_h(x) + y_p(x)$. \\
Autrement dit, on trouve toutes les solutions en ajoutant une solution particulière ($y_p$) aux solutions de l'équation homogène ($y_h$).}




Comme on sait résoudre l'équation homogène $y'=ay$ alors la recherche de la solution générale de $(E)$ se réduit donc à la recherche d'une solution particulière.
Pour trouver cette solution particulière des indications vous seront fournies.



%%%%%%%%%%%%%%%%%%%%%%%%%%%%%%%%%%%%%%%%%%%%%%%%%%%
\subsection*{Exemples}

\emph{Exemple 1.}
Résoudre l'équation $(E)$ : $y' = 2y + 7$ en cherchant une solution particulière sous la forme d'une fonction constante.
\begin{itemize}
  \item \emph{\'Equation homogène.} L'équation homogène est $(E_h)$ : $y'=2y$, les solutions sont les $y_h(x) = Ce^{2x}$ pour chaque $C \in \Rr$.

  \item \emph{Solution particulière.} Cherchons une solution constante de $(E)$. Soit $y_p(x)=k$, alors $y_p'(x) = 0$, donc l'équation $(E)$ devient $0=2k+7$, ce qui implique $k=-\frac72$. Une solution particulière est donc $y_p(x)=-\frac72$.

  \item \emph{Solutions générales.} Les solutions générales de $(E)$ sont donc  les fonctions de la forme "les solutions de l'équation homogène + une solution particulière" :
$$y(x) = y_h(x) + y_p(x),$$
c'est-à-dire pour notre exemple
$$y(x) = Ce^{2x} - \frac72 \qquad \text{ avec } C \in \Rr.$$
Chaque valeur de la constante $C$ donne une solution.
\end{itemize}

\bigskip

\emph{Exemple 2.}
Résoudre l'équation $(E)$ : $y'+y = x^2$ en cherchant une solution particulière sous la forme d'une fonction polynomiale de degré 2, $y_p(x) = ax^2+bx+c$.
\begin{itemize}
  \item \emph{\'Equation homogène.} L'équation homogène est $(E_h)$ : $y'+y=0$, c'est-à-dire $y'=-y$, dont les solutions sont les $y_h(x) = Ce^{-x}$ pour chaque $C \in \Rr$.

  \item \emph{Solution particulière.} Cherchons une solution à $(E)$ sous la forme polynomiale indiquée : $y_p(x)=ax^2+bx+c$. On a alors $y_p'(x) = 2ax+b$, donc l'équation $(E)$ devient $(2ax+b)+(ax^2+bx+c)=x^2$. Pour trouver $a,b,c$ on regroupe les coefficients devant $x^2$, $x$, et $1$ pour ensuite effectuer une identification :
  $$a \cdot x^2 + (2a+b) \cdot x + (b+c) \cdot 1 = x^2$$
  ce qui donne 
  $$a \cdot x^2 + (2a+b) \cdot x + (b+c)\cdot 1 = x^2 = 1 \cdot x^2 + 0 \cdot x + 0 \cdot 1.$$
  On identifie les coefficients du polynômes de gauche avec les coefficients du polynôme de droite afin d'obtenir :
  $$a=1 \qquad 2a+b = 0 \qquad b+c = 0$$
  et donc 
  $$a = 1 \qquad b=-2 \qquad c = 2.$$
  Ainsi une solution particulière est :
$$y_p(x) = x^2-2x+2.$$
  

  \item \emph{Solutions générales.} Les solutions générales de $(E)$ sont donc 
 les fonctions  :
$$y(x) = y_h(x) + y_p(x),$$
c'est-à-dire
$$y(x) = Ce^{-x} + x^2-2x+2 \qquad \text{ avec } C \in \Rr.$$
\end{itemize}


\end{document}
