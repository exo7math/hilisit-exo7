\documentclass[11pt,class=report,crop=false]{standalone}
\usepackage{exo7hilisit}

\begin{document}

%%%%%%%%%%%%%%%%%%%%%%%%%%%%%%%%%%%%%%%%%%%%%%%%%%%%%%%%%%%%%%%%%%%%%%
%%%%%%%%%%%%%%%%%%%%%%%%%%%%%%%%%%%%%%%%%%%%%%%%%%%%%%%%%%%%%%%%%%%%%%


\entete{Hilisit}{Capacité mathématiques}

\titre{\'Equations différentielles -- Partie 2 : Notion d'équation différentielle} 



\encadre{
	\emph{Savoir.}
	\begin{itemize}[label=$\square$]
		\item Comprendre ce qu'est une équation différentielle.
%		\item Savoir expliquer les termes d'une équation différentielle à partir des notions d'effectif, de taux de croissance ou de proportionnalité.
	\end{itemize}
	\emph{Savoir-faire.}
	\begin{itemize}[label=$\square$]
		\item Savoir vérifier qu'une fonction est solution d'une équation différentielle.
%		\item Savoir déterminer les solutions constantes d'une équation différentielle. 		
 %       \item Savoir trouver l'équation différentielle associée à une situation décrite par un texte.
	\end{itemize}
}


\bigskip

%%%%%%%%%%%%%%%%%%%%%%%%%%%%%%%%%%%%%%%%%%%%%%%%%%%



Nous nous intéressons à des équations où l'inconnue à trouver n'est pas un nombre mais une fonction. Par exemple, considérons l'équation $f'(x)=f(x)$ pour tout $x\in\Rr$. On cherche toutes les fonctions $f$ possibles satisfaisant cette équation, c'est-à-dire qui sont égales à leur propre dérivée. Vous en connaissez au moins une... Laquelle ? La fonction exponentielle ! Il existe d'autres solutions. En fait, les solutions de cette équation sont les fonctions de la forme $f(x)=Ce^{x}$ où $C$ est une constante réelle. 


%%%%%%%%%%%%%%%%%%%%%%%%%%%%%%%%%%%%%%%%%%%%%%%%%%%
\subsection*{Définition d'une équation différentielle}

 On appelle \textbf{équation différentielle} toute équation, où l'inconnue est une fonction $f$, mettant en relation $f$ et $f'$ (et éventuellement les dérivées successives $f''$, $f'''$, \dots).
 
 \emph{Exemples.}
 Les équations suivantes sont des exemples d'équations différentielles:
 \begin{itemize}
 	\item[] $f'(x)=e^xf(x)+x$,
 	\item[] $f''(x)=-f'(x)+2$,
 	\item[] $f(x)f'(x)=-\ln(f(x))$.
 \end{itemize}

 \emph{Notation.} Il faut s'habituer aux notations variées pour une équation différentielle.
 Voici différentes notations de la même équation :
 \begin{itemize}
 	\item[] $f'(x) = -f(x)$ \qquad (fonction inconnue $f$ de variable $x$),
 	\item[] $y'(x) = -y(x)$ \qquad (fonction inconnue $y$ de variable $x$),
 	\item[] $y'(t) = -y(t)$ \qquad (fonction inconnue $y$ de variable $t$),
 	\item[] $y' = -y$ \qquad \qquad fonction inconnue $y$ : c'est cette dernière notation que nous adoptons, le nom de la variable sera $x$ même s'il n'est pas spécifié dans l'équation.   
 \end{itemize}

  \emph{Exercice.} Trouver/deviner une solution (ou mieux plusieurs) des équations différentielles suivantes :
 \begin{itemize}
 	\item[] $y' = -y$
 	\item[] $y' = \sin(2x)$
  	\item[] $y'(x) = 3y(x)$  
  	\item[] $y''(x) = y(x)$  
 \end{itemize}

\emph{Remarque.} Trouver une primitive d'une fonction $f$, c'est en fait résoudre l'équation différentielle $y'=f$ où l'inconnue est $y$ (appelée $F$ en section précédente) et $f$ est donnée.


%%%%%%%%%%%%%%%%%%%%%%%%%%%%%%%%%%%%%%%%%%%%%%%%%%%
\subsection*{Solutions particulières} 

Résoudre une équation différentielle c'est trouver toutes les fonctions qui satisfont l'équation. En général, c'est un problème très difficile, voire même impossible !

Nous nous placerons dans deux situations plus simples :
\begin{itemize}
  \item vérifier qu'une fonction donnée est bien solution d'une équation différentielle,
  \item déterminer les solutions constantes d'une équation différentielle.
\end{itemize}

\paragraph*{Exemple 1} 
\begin{equation*}
\label{EquExZero}
y' = 2y + 4x
\end{equation*}
Il s'agit donc de trouver des fonctions $f$ telles que $f'(x)=2f(x)+4x$, pour tout $x\in\Rr$.

\begin{itemize}
  \item Vérifier que $f(x) = -2x-1$  est solution.
  \item Vérifier que $f(x) = \exp(2x)-2x-1$  est aussi solution.
  \item Plus généralement vérifier que $f(x) = C\exp(2x)-2x-1$  est solution (quel que soit $C\in\Rr$).
\end{itemize}

Les deux premiers points sont des cas particuliers du troisième (avec $C=0$ ou $C=1$). Faisons le cas général. Soit $f(x) = C\exp(2x)-2x-1$. On va calculer 
ce qui correspond au terme de gauche, puis au terme de droite de l'équation $y' = 2y + 4x$ :
\begin{itemize}
  \item on calcule d'abord la dérivée $f'$ : \quad  $f'(x) = 2C\exp(2x)-2$,
  \item puis $2f(x)+4x$ : 
$$2f(x)+4x = 2 \big( C\exp(2x)-2x-1 \big ) +4x = 2C\exp(2x) -2.$$
\end{itemize}
On a donc prouvé que $f'(x) = 2f(x)+4x$ (pour tout $x \in \Rr$), donc notre fonction $f$ est bien solution de l'équation différentielle $y' = 2y + 4x$, et ceci quelle que soit la constante $C \in \Rr$.


 
\paragraph*{Exemple 2} On considère l'équation différentielle 
\begin{equation}
\label{EquExDeux}
y'(x)=y(x)^2-1
\tag{$E$}
\end{equation}
Déterminons les solutions constantes de cette équation différentielle. Pour cela, rappelons les points suivants:
 \begin{itemize}
 	\item Une fonction définie et dérivable sur un intervalle $I$ est constante si et seulement si sa dérivée est nulle sur $I$.
 	\item Pour connaître une fonction $f$ constante sur un intervalle $I$, il suffit de la connaître la valeur en un point $x_0\in I$. 
 \end{itemize}
Considérons une fonction constante $f(x)=k$ (pour tout $x$), alors on sait que  $f'(x)=0$ (pour tout $x$). Si en plus $f$ est solution de (\ref{EquExDeux})
alors on doit résoudre l'équation réelle (équation dont l'inconnue est un réel que nous noterons $k$) 
 \begin{equation*}
\label{EquConst}
	0=k^2-1.
 \end{equation*}
 Les deux solutions réelles sont  $k=1$  ou $k=-1$. Au final, l'équation différentielle (\ref{EquExDeux}) possède deux solutions constantes : $f(x)=1$ et $f(x)=-1$. Notons que l'équation possède peut-être aussi des solutions non-constantes que nous n'avons pas déterminées. 
 
 Ajoutons enfin qu'une équation différentielle ne possède pas nécessairement de solutions constantes. Essayez par exemple de trouver des solutions constantes à l'équation différentielle $y' +2y = 4x^2$ : vous n'y parviendrez pas ! 



%%%%%%%%%%%%%%%%%%%%%%%%%%%%%%%%%%%%%%%%%%%%%%%%%%%
\subsection*{Tangente (rappels)}



\begin{center}
\begin{tikzpicture}[scale=2]

	\draw[->,>=latex, gray, very thin] (-0.5,0) -- (3.3,0);
	\draw[->,>=latex, gray, very thin] (0,-0.5) -- (0,2.8);

%	\draw[domain=-0.25:2.5,black,thick,smooth] plot (\x,{0.6+0.4*\x+0.6*cos(4*\x r)});

    \draw[domain=0:2.35, blue,very thick,smooth] plot (\x,{2-(\x-1)^2)});


   \def \x{0.7}
    \coordinate (A) at ({\x},{2-(\x-1)^2)});
%    \fill (A) circle (1.5pt) node[above] {$M_0$};

    \draw[myred,thick] (A)--+(1.5,{1.5*(2-2*\x)}) node[below]{$T$};
    \draw[myred,thick] (A)--+(-1.5,{-1.5*(2-2*\x)}) ;

  \draw[dashed] (A -| 0,0) node[left]{$f(x_0)$} -- (A)--({\x},0) node[below]{$x_0$};

%\foreach \i in {4,3,...,1}
%{
%  \def\xx{\x + 1.5-0.3*\i};
%    \coordinate (M) at ({\xx},{2-(\xx-1)^2)});
%    \fill (M) circle (1.5pt);
%    \draw (A)--(M)--+($\i*(M)-\i*(A)$)--(A)--+($\i*(A)-\i*(M)$);
%    \coordinate (P) at ({\xx},0);
%};
%  \draw[dashed] (M)--(P) node[below]{$x$};
%  \node[above right] at (M) {$M$};

\end{tikzpicture}
\end{center}




La dérivée en $x_0$ d'une fonction $f$ est le coefficient directeur de la tangente au point $(x_0,f(x_0))$ du graphe de $f$.

L'équation de cette tangente est :
\mybox{$y = (x-x_0) f'(x_0) + f(x_0)$}


Exemple : quelle est l'équation de la tangente au graphe de $f(x)=e^{2x}$ en $x_0=1$ ?
On a $f'(x) = 2e^{2x}$, $f(x_0)=f(1)=e^2$, $f'(x_0)=f'(1) = 2e^2$.
L'équation de la tangente est $y = (x-1)2e^2 + e^2$, ce qui s'écrit aussi
$y = 2e^2x - e^2$.



%%%%%%%%%%%%%%%%%%%%%%%%%%%%%%%%%%%%%%%%%%%%%%%%%%%
\subsection*{\'Equation différentielle et tangente}

Une équation différentielle donne une relation entre une fonction solution et sa dérivée. On peut ainsi obtenir des informations sur la tangente au graphe de cette solution, parfois on n'a même pas besoin de calculer explicitement la solution.


\emph{Exemple.}
On considère l'équation $(E)$ : $y' = x y + 1$ (que l'on ne cherchera pas à résoudre).
\begin{itemize}
  \item \emph{Quelle est l'équation de la tangente en $x=0$ au graphe de la solution $f$ qui vérifie $f(0)=1$ ?} 

  On sait que cette tangente passe par le point $(0,1)$ car $f(0)=1$, mais on doit en plus connaître le coefficient directeur qui est donné par $f'(0)$.

  Comme $f$ est solution de l'équation différentielle $(E)$ alors $f'(x) = x f(x) + 1$, pour tout $x\in\Rr$. En particulier pour $x=0$ on obtient $f'(0) = 0 \cdot f(0) + 1$, donc $f'(0)=1$. Ainsi la tangente cherchée passe par le point $(0,1)$ et a pour pente $1$, c'est donc la droite d'équation $y = x+1$.

  \item \emph{Quelle est l'équation de la tangente en $x=2$ au graphe de la solution $g$ qui vérifie $g(2)=3$ ?} 

  Comme $g(2)=3$ cette tangente passe par le point $(2,3)$.
  Comme $g$ est solution de l'équation différentielle $(E)$ alors $g'(x) = x g(x) + 1$, pour tout $x\in\Rr$. En particulier pour $x=2$ on obtient $g'(2) = 2 \cdot g(2) + 1$, donc $g'(2)=7$. Ainsi la tangente cherchée passe par le point $(2,3)$ et a pour pente $7$, c'est donc la droite d'équation $y = 7(x-2)+3$, qui s'écrit aussi $y=7x-11$.
\end{itemize}

Notez que l'on n'a pas calculé explicitement ni $f(x)$, ni $g(x)$.


%%%%%%%%%%%%%%%%%%%%%%%%%%%%%%%%%%%%%%%%%%%%%%%%%%%
\subsection*{Modélisation}


Le concept d'équation différentielle intervient dans de nombreux domaines scientifiques, entre autres :
\begin{itemize}
	\item En biologie avec l'étude d'une population (comme la population de micro-organismes) où l'on connaît des règles pour décrire sa croissance (comme le taux de natalité/mortalité).

	\item En physique avec notamment la loi fondamentale de la mécanique qui relie l'accélération à la somme des forces. Cela conduit à une équation différentielle car l'accélération est la dérivée de la vitesse, et donc la dérivée seconde de la position. L'étude du mouvement des corps célestes en astronomie, ainsi que la physique quantique (avec la célèbre équation de Schr\"odinger) sont également des domaines où les équations différentielles sont omniprésentes.

	\item En radioactivité avec l'étude de la désintégration de noyaux radioactifs et le calcul de la demi-vie radioactive qui permet en particulier la datation des matières organiques anciennes.
 \end{itemize}


\medskip

\emph{Exemple.}
	\emph{"On étudie la population de chenilles qui s'est introduite dans un groupe d'arbres. On note $N(t)$ le nombre de chenilles au cours du temps. Des mesures effectuées montrent que le taux de croissance des chenilles au temps $t$ est de $4\%$ de la population."}
	
	Ce texte signifie que la variation du nombre de chenilles au temps $t$ (c'est-à-dire la dérivée du nombre de chenilles au temps $t$, donc $N'(t)$) est proportionnelle à l'effectif des chenilles au temps $t$ (c'est-à-dire proportionnelle à $N(t)$) et que le coefficient de proportionnalité vaut $0.04$ (la population est croissante donc le coefficient est positif).
	
	L'équation différentielle associée au problème est donc:
	\[N'(t)=0,04 \; N(t).\]

   Avec nos notations habituelles cette équation différentielle est : $y' = 0,04  y$.


\end{document}
