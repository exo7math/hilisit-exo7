\documentclass[11pt,class=report,crop=false]{standalone}
\usepackage{exo7hilisit}

\begin{document}


\entete{Hilisit}{Capacité mathématiques}

\titre{\'Equations différentielles -- Partie 3 : $y'=ay$}

\bigskip
\bigskip


%%%%%%%%%%%%%%%%%%%%%%%%%%%%%%%%%%%%%%%%%%%%%%%%%%%%%%%%%%%%
%\section{$y'=ay$}

\exercice{}
\enonce
Résoudre les équations différentielles suivantes, c'est-à-dire trouver toutes les fonctions solutions.
\begin{enumerate}
  \item $y'=3y$
  \item $y'+2y=0$
  \item $4y'-5y=0$
\end{enumerate} 
\finenonce

\finexercice

\exercice{}
\enonce
Trouver la solution des équations différentielles suivantes vérifiant la condition initiale donnée.
\begin{enumerate}
  \item $y'=-y$ avec $y(0)=2$
  \item $y'=3y$ avec $y(0)=-1$
  \item $2y'=y$ avec $y(2)=3$
\end{enumerate} 
\finenonce

\finexercice


\exercice{}
\enonce
Dire si les affirmations suivantes sont vraies ou fausses. Justifier votre réponse par un résultat du cours ou un contre-exemple.
 \begin{enumerate}
  \item \og{}Les solutions de $y'=2y$ sont toutes des fonctions croissantes.\fg{}
  \item \og{}L'équation différentielle $y'=-y$ admet une seule solution constante.\fg{}
  \item \og{}Il existe une unique solution à l'équation différentielle $y'=7y$ qui vérifie $y(0)>0$.\fg{}
  \item \og{}La solution de l'équation différentielle $y'=-2y$ qui vérifie $y(0)=-1$ tend vers $-\infty$ en $+\infty$.\fg{}
\end{enumerate} 
\finenonce

\finexercice


\exercice{}
\enonce
\`A la mort d'un être vivant, le nombre $N(t)$ de noyaux radioactifs de carbone 14 (en milliards) décroît selon l'équation différentielle
$$N'(t) = -k N(t)$$
où $k>0$ est une constante.

On cherche à dater un os d'animal trouvé dans une grotte.

On prend pour temps d'origine $t=0$, la date de mort de l'animal. L'unité de temps est l'année. On sait que la \og{}demi-vie\fg{} du carbone 14 est 5500 ans, cela signifie qu'à chaque période de 5500 années, la moitié des noyaux se sont désintégrés.

On sait que le nombre de noyaux lors du vivant de l'animal était $N_0=16$ (ce nombre reste constant pour tous les animaux  vivants et commence à décroître à leur mort). On mesure $N_1 = 0,20$ le nombre de noyaux dans l'os au temps présent.
\begin{enumerate}
  \item Résoudre l'équation différentielle en fonction de $N_0$ et de $k$.
  \item La période de demi-vie donne la relation $N(5500)=\frac{N_0}{2}$. Calculer alors la valeur de la constante $k$.
  \item Combien d'années auparavant cet animal a-t-il vécu ?
\end{enumerate} 
\finenonce

\finexercice

\end{document}
