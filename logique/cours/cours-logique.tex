\documentclass[11pt,a4paper]{report}
\usepackage{exo7hilisit}


\begin{document}

%%%%%%%%%%%%%%%%%%%%%%%%%%%%%%%%%%%%%%%%%%%%%%%%%%%%%%%%%%%%%%%%%%%%%%
%%%%%%%%%%%%%%%%%%%%%%%%%%%%%%%%%%%%%%%%%%%%%%%%%%%%%%%%%%%%%%%%%%%%%%

\entete{Hilisit}{Capacité mathématiques}

\titre{Cours -- Logique, ensembles et raisonnements}

\bigskip
\bigskip

\begin{quote}
\center
\emph{
    Les mathématiques sont un formidable assemblage de propositions qui s'enchaînent les unes avec les autres d'une manière logique. Une proposition (que l'on peut tout à fait nommer \emph{propriété}, ou encore \emph{théorème}) n'est adoptée au sein des théories mathématiques que si l'on a rigoureusement démontré qu'elle est vraie.     
    Cela pose plusieurs questions : qu'est-ce que la \emph{logique} qui permet de comprendre les propositions et de les lier les unes aux autres ? Comment démontrer qu'une proposition est vraie ou fausse ?
    }
\end{quote}

\bigskip
\bigskip


\textbf{Sections}
\begin{enumerate}[label=\arabic*.]
    \item \textbf{Logique}
    
    Thèmes : 
    Comprendre les connecteurs logiques  ET, OU et NON.
    Comprendre les quantificateurs $\forall$ et $\exists$.
    Comprendre l'implication, l'équivalence, la contraposée.
    
    
    Objectifs :
    Savoir décider si une assertion est vraie ou fausse.
    Savoir écrire des négations simples.
    
    
    
     \item  \textbf{Ensembles}
 
     Thèmes :
     Maîtriser le vocabulaire des ensembles : élément, inclusion, complémentaire, union, intersection.
     Maîtriser le vocabulaire des fonctions : image, antécédent, bijection.
 
     Objectifs :    
     Savoir calculer des unions, intersections, complémentaires d'ensembles.
     Savoir déterminer un domaine de définition.
     Savoir calculer une composition et montrer qu'une fonction est bijective.
    
    
    \item  \textbf{Raisonnements}
    
    Thèmes et objectifs :     
    Démonstration au cas par cas.
    Raisonnement par l'absurde.
    Preuve par contraposition.
    Récurrence.  

\end{enumerate}


\bigskip
\bigskip



\vfill

\begin{center}
\begin{minipage}{0.8\textwidth}
\center
Auteur : Barnabé Croizat de l'université de Lille.

Relecture : Arnaud Bodin, Christine Sacré et Abdelkader Necer.

  \medskip
  
Ce travail a été effectué en 2021-2022 dans le cadre d'un projet Hilisit porté  Unisciel.
\end{minipage}

  \medskip

\raisebox{1ex}{\includegraphics[height=1.8cm]{logo-unisciel}}\qquad\qquad
\includegraphics[height=2.2cm]{logo-ulille}

  \medskip
  
Ce document est diffusé sous la licence \emph{Creative Commons -- BY-NC-SA -- 4.0 FR}.


Sur le site Exo7 vous pouvez récupérer les fichiers sources.

\vspace*{0cm}

\end{center}


\newpage


%%%%%%%%%%%%%%%%%%%%%%%%%%%%%%%%%%%%%%%%%%%%%%%%%%%%%%%%%%%%%%%%%%%%%%
%%%%%%%%%%%%%%%%%%%%%%%%%%%%%%%%%%%%%%%%%%%%%%%%%%%%%%%%%%%%%%%%%%%%%%

\import{./}{logique-part1.tex}
\newpage

\import{./}{logique-part2.tex}
\newpage

\import{./}{logique-part3.tex}
\newpage


%%%%%%%%%%%%%%%%%%%%%%%%%%%%%%%%%%%%%%%%%%%%%%%%%%%%%%%%%%%%%%%%%%%%%%
%%%%%%%%%%%%%%%%%%%%%%%%%%%%%%%%%%%%%%%%%%%%%%%%%%%%%%%%%%%%%%%%%%%%%%

\end{document}
