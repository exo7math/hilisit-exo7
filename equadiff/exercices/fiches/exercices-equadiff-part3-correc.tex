\documentclass[11pt,class=report,crop=false]{standalone}
\usepackage{exo7hilisit}

\begin{document}


\entete{Hilisit}{Capacité mathématiques}

\titre{\'Equations différentielles -- Partie 3 : $y'=ay$}

\bigskip
\bigskip



%%%%%%%%%%%%%%%%%%%%%%%%%%%%%%%%%%%%%%%%%%%%%%%%%%%%%%%%%%%%
%\section{$y'=ay$}

\exercice{}
\enonce
Résoudre les équations différentielles suivantes, c'est-à-dire trouver toutes les fonctions solutions.
\begin{enumerate}
  \item $y'=3y$
  \item $y'+2y=0$
  \item $4y'-5y=0$
\end{enumerate} 
\finenonce

\indication
Les solutions de $y'=ay$ sont les fonctions $y(x)=Ce^{ax}$ où $C$ est une constante.
\finindication

\correction
\sauteligne
\begin{enumerate}
  \item $y'=3y$ : les solutions sont $y(x)=Ce^{3x}$ où $C$ est une constante.
  \item $y'+2y$ équivaut à $y'=-2y$ : les solutions sont $y(x)=Ce^{-2x}$ où $C$ est une constante.
  \item $4y'-5y=0$ équivaut à $y'=\frac54y$  : les solutions sont $y(x)=Ce^{\frac54x}$ où $C$ est une constante.
\end{enumerate}
\fincorrection
\finexercice

\exercice{}
\enonce
Trouver la solution des équations différentielles suivantes vérifiant la condition initiale donnée.
\begin{enumerate}
  \item $y'=-y$ avec $y(0)=2$
  \item $y'=3y$ avec $y(0)=-1$
  \item $2y'=y$ avec $y(2)=3$
\end{enumerate} 
\finenonce

\indication
Les solutions de $y'=ay$ sont les $y(x)=Ce^{ax}$ où $C$ est une constante. 
Mais ici, il faut en plus déterminer la constante $C$ afin que $y(x_0)=y_0$ ($x_0$ et $y_0$ correspondant à la condition initiale imposée).
\finindication

\correction
\sauteligne
\begin{enumerate}
  \item $y'=-y$ : les solutions sont $y(x)=Ce^{-x}$ où $C$ est une constante.
  On veut en plus $y(0)=2$, donc $Ce^{-0}=2$, comme $e^0=1$ alors $C=2$. L'unique solution cherchée est $y(x)=2e^{-x}$.

  \item $y'=3y$ : les solutions sont $y(x)=Ce^{3x}$ où $C$ est une constante.
  On veut en plus $y(0)=-1$, donc $Ce^{3\cdot 0}=-1$, donc $C=-1$. L'unique solution cherchée est $y(x)=-e^{3x}$.

  \item $2y'=y$ : les solutions sont $y(x)=Ce^{\frac12x}$ où $C$ est une constante.
  On veut en plus $y(2)=3$, donc $Ce^{\frac12 \cdot 2}=3$, donc $C e^1=3$ d'où $C = \frac{3}{e}$. L'unique solution cherchée est $y(x)=\frac{3}{e}e^{\frac12x}$ que l'on peut écrire aussi $y(x)=3e^{\frac12x-1}$.
\end{enumerate}
\fincorrection
\finexercice


\exercice{}
\enonce
Dire si les affirmations suivantes sont vraies ou fausses. Justifier votre réponse par un résultat du cours ou un contre-exemple.
 \begin{enumerate}
  \item \og{}Les solutions de $y'=2y$ sont toutes des fonctions croissantes.\fg{}
  \item \og{}L'équation différentielle $y'=-y$ admet une seule solution constante.\fg{}
  \item \og{}Il existe une unique solution à l'équation différentielle $y'=7y$ qui vérifie $y(0)>0$.\fg{}
  \item \og{}La solution de l'équation différentielle $y'=-2y$ qui vérifie $y(0)=-1$ tend vers $-\infty$ en $+\infty$.\fg{}
\end{enumerate} 
\finenonce

\noindication

\correction
\sauteligne
 \begin{enumerate}
  \item Faux. Les solutions de $y'=2y$ sont les $y(x)=Ce^{2x}$ où $C$ est une constante réelle. Par exemple pour $C=-1$, la solution $y(x)=-e^{2x}$ est décroissante.
  \item Vrai. La seule solution constante est $y(x)=0$ (obtenue via l'expression générale $y(x) = C e^{-x}$ pour $C=0$).
  \item Faux. Il existe une solution pour chaque valeur de $C=y(0)$. Par exemple pour $C=1$, $y(x)=e^{7x}$ est solution et pour $C=2$, $y(x)=2e^{7x}$ est aussi solution.
  \item Faux. La solution cherchée est $y(x)=-e^{-2x}$ qui tend vers $0$ lorsque $x$ tend vers $+\infty$.
\end{enumerate} 
\fincorrection
\finexercice


\exercice{}
\enonce
\`A la mort d'un être vivant, le nombre $N(t)$ de noyaux radioactifs de carbone 14 (en milliards) décroît selon l'équation différentielle
$$N'(t) = -k N(t)$$
où $k>0$ est une constante.

On cherche à dater un os d'animal trouvé dans une grotte.

On prend pour temps d'origine $t=0$, la date de mort de l'animal. L'unité de temps est l'année. On sait que la \og{}demi-vie\fg{} du carbone 14 est 5500 ans, cela signifie qu'à chaque période de 5500 années, la moitié des noyaux se sont désintégrés.

On sait que le nombre de noyaux lors du vivant de l'animal était $N_0=16$ (ce nombre reste constant pour tous les animaux  vivants et commence à décroître à leur mort). On mesure $N_1 = 0,20$ le nombre de noyaux dans l'os au temps présent.
\begin{enumerate}
  \item Résoudre l'équation différentielle en fonction de $N_0$ et de $k$.
  \item La période de demi-vie donne la relation $N(5500)=\frac{N_0}{2}$. Calculer alors la valeur de la constante $k$.
  \item Combien d'années auparavant cet animal a-t-il vécu ?
\end{enumerate} 
\finenonce

\noindication

\correction
\sauteligne
\begin{enumerate}
  \item La solution de $N'(t) = -k N(t)$ est $N(t) = Ce^{-kt}$, or $N(0)=N_0=C$ donc la solution est $N(t) = N_0e^{-kt}$ autrement dit $N(t) = 16e^{-kt}$.

  \item $N(5500)=\frac{16}{2}=8$, donc $16e^{-k \cdot 5500}=8$. Cela donne $e^{-k \cdot 5500} = \frac12$. On compose par le logarithme de chaque côté : 
$\ln\big( e^{-k \cdot 5500} \big) = \ln(\frac12)$, donc $-k \cdot 5500 = -\ln(2)$. Conclusion : $k = \frac{\ln(2)}{5500} \simeq 0,000126$ et la solution est $N(t) \simeq 16 e^{-0,000126 \cdot t}$.

  \item Notons $\tau$ le temps écoulé depuis la mort de l'animal. Alors on sait que $N(\tau)=N_1$, donc $ 16 e^{-0,000126 \cdot \tau} = 0,2$. Cela donne
  $e^{-0,000126 \cdot \tau} = 0,0125$ ; donc en composant par le logarithme 
  $-0,000126 \cdot \tau = \ln(0,0125)$ ainsi $\tau \simeq 34777$. On peut donc dire que l'animal a vécu il y a environ 35 000 années. 
\end{enumerate} 
\fincorrection
\finexercice

\end{document}
