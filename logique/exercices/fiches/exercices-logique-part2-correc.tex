\documentclass[11pt,class=report,crop=false]{standalone}
\usepackage{exo7hilisit}

\begin{document}


\entete{Hilisit}{Capacité mathématiques}

\titre{Logique, ensembles et raisonnements -- Partie 2}

\bigskip
\bigskip


\exercice{}
\enonce
Remplacer les pointillés par le symbole le plus adapté parmi $\in$, $\notin$, $\subset$, $\supset$.

\begin{enumerate}
    \item $ [3,5] \qquad \ldots\ldots \qquad \{ x \in \Rr \mid 2 \le x \le 7 \} $
    \item $ 2 \qquad \ldots\ldots \qquad \{ x \in \Rr \mid x^2 \ge 5 \} $
    \item $\pi = 3.14\dots  \qquad \ldots\ldots \qquad \Rr \setminus \Qq $
    \item $[1,9] \qquad \ldots\ldots \qquad [1,4] \cup [5,9] $
    \item $\{ 0 \} \qquad \ldots\ldots \qquad \Rr_+$
    \item $0 \qquad \ldots\ldots \qquad \Zz \setminus \Nn $   
    \item $[-7,5] \cap [-2,8] \qquad \ldots\ldots \qquad [-1,1] $  
\end{enumerate}    
\finenonce

\indication
Rappels :
$\in$ "appartient à" ; $\notin$ "n'appartient pas à" sont utilisés pour des éléments, $\subset$ "est contenu dans", $\supset$ "contient" sont utilisés pour des ensembles.
\finindication

\correction
\sauteligne
\begin{enumerate}
    \item $ [3,5] \subset \{ x \in \Rr \mid 2 \le x \le 7 \} $ car on rappelle que 
    $ [3,5] = \{ x \in \Rr \mid 3 \le x \le 5 \} $.

    \item $ 2 \notin \{ x \in \Rr \mid x^2 \ge 5 \} $
    
    \item $\pi  \in  \Rr \setminus \Qq $
    
    \item $[1,9] \supset [1,4] \cup [5,9] $
    
    \item $\{ 0 \} \subset \Rr_+$ (et pas "$\in$" car $\{ 0 \}$ est un ensemble, ce n'est pas l'élément $0$). 

    \item $0 \notin \Zz \setminus \Nn $ ($0 \in \Nn$ donc on le retire ici de $\Zz$).
    
    \item $\big([-7,5] \cap [-2,8]\big) \supset [-1,1]$  
\end{enumerate}  
\fincorrection
\finexercice



\exercice{}
\enonce
Déterminer le domaine de définition de la fonction $x \mapsto f(x)$ dans chacun des cas suivants :
\begin{enumerate}
    \item $f(x) = \sqrt{-x+3}$
    \item $f(x) = \frac{1}{x} + \frac{1}{x^2-1}$
    \item $f(x) = \exp(x^2+1)$
    \item $f(x) = \ln( 5x + 8 )$
    \item $f(x) = \ln\big( (x-1)(x+2) \big)$
    \item $f(x) = \sqrt{ x^2+3x-2 }$
\end{enumerate} 
\finenonce

\indication
L'expression $\sqrt{x}$ est définie pour $x\ge0$.
L'expression $\frac1x$ est définie pour $x\neq0$.
L'expression $\ln(x)$ est définie pour $x>0$.
\finindication

\correction
On désigne par $\mathcal{D}_f$ l'ensemble de définition de $f$.
\begin{enumerate}
    \item $f(x) = \sqrt{-x+3}$.
    On doit avoir $-x+3\ge0$, c'est-à-dire $3 \ge x$, donc $\mathcal{D}_f = ]-\infty,3]$.
    
    \item $f(x) = \frac{1}{x} + \frac{1}{x^2-1}$.
    Les dénominateurs ne doivent pas s'annuler.
    Les dénominateurs s'annulent lorsque $x = 0$ ou $x^2-1 = 0$, c'est-à-dire 
    $x \in \{0,1,-1\}$. 
    Ainsi 
    $$\mathcal{D}_f = \Rr \setminus \{0,1,-1\} = ]-\infty,-1[ \cup ]-1,0[ \cup ]0,1[ \cup ]1,+\infty[.$$
    
    
    \item $f(x) = \exp(x^2+1)$.
    Cette expression est définie pour tout $x$ réel :  $\mathcal{D}_f = \Rr$.
    
    \item $f(x) = \ln( 5x + 8 )$.
    On doit avoir $5x+8 > 0$, c'est-à-dire $x > -\frac85$, donc $\mathcal{D}_f = ]-\frac85,+\infty[$.
    
    \item $f(x) = \ln\big( (x-1)(x+2) \big)$.
    
    On étudie le signe de $(x-1)(x+2)$ (tableau de signes immédiat plutôt que développement et étude du signe d'un trinôme selon ses racines), qui est strictement positif pour $x>1$ et aussi pour $x<-2$.
    Donc $\mathcal{D}_f = ]-\infty,-2[ \cup ]1,+\infty[$.
    
    \item $f(x) = \sqrt{ x^2+3x-2 }$.
    On étudie le signe de $x^2+3x-2$. Pour cela on cherche d'abord les solutions de $x^2+3x-2=0$.
    Les racines sont  $x_1 = \frac{-3 - \sqrt{17}}{2}$ et $x_2 = \frac{-3 + \sqrt{17}}{2}$.
    Le trinôme $x^2+3x-2$ est positif à l'extérieur des racines, donc 
    $\mathcal{D}_f = ]-\infty,x_1] \cup [x_2,+\infty[$.
    
\end{enumerate} 
\fincorrection
\finexercice



\exercice{}
\enonce
Soient $f,g : \Rr \to \Rr$ définies par $f(x)=2x+1$ et $g(x)=x^2-3x$.
\begin{enumerate}
    \item Déterminer l'expression de la fonction $f \circ g$.
    \item Déterminer l'expression de la fonction $g \circ f$.
    \item Montrer que $(g\circ f)(\frac{-1}{2})=0$ et en déduire, pour $x\in \Rr$, la factorisation de l'expression $(g \circ f)(x)$.      
\end{enumerate} 
\finenonce

\indication
$(f \circ g)(x) = f \big( g(x) \big)$ \\
$(g \circ f)(x) = g \big( f(x) \big)$ \\
Si un polynôme $P(X)$ s'annule en $X=\alpha$, alors il se factorise par $(X-\alpha)$.
\finindication

\correction
\sauteligne
\begin{enumerate}
    \item $f \circ g$
    $$(f \circ g)(x) = f \big( g(x) \big) = f(x^2-3x) = 2(x^2-3x) + 1 = 2x^2-6x + 1.$$
    
    
    \item $g \circ f$
$$(g \circ f)(x) = g \big( f(x) \big) = g(2x+1) = (2x+1)^2 -3(2x+1) = (4x^2+4x+1) - (6x+3) = 4x^2-2x -2.$$

   \item Calculons  $(g \circ f)( \tfrac{-1}{2} )$.
   
   D'une part $f(-\tfrac12) = 2 (-\tfrac12) + 1 = 0$. 
   Donc $(g \circ f)( \tfrac{-1}{2} ) = g(0) =0^2 - 3 \times 0 = 0$.
   
   D'autre part, on a calculé que $(g \circ f)(x) = 4x^2-2x -2$. Donc on vient de prouver que 
   $-\tfrac 12 $ est une racine de $4x^2-2x -2$. 
   On en déduit $4x^2-2x -2 = a(x+\tfrac12)(x-b)$. Par identification on trouve $a=4$ et $b=1$.
   Conclusion : $4x^2-2x -2 = 4(x+\tfrac12)(x-1)$.
\end{enumerate}
\fincorrection
\finexercice


\exercice{}
\enonce
On veut déterminer la bijection réciproque de la fonction $f$ définie par :
$$f(x) = \frac{2x-1}{x-3}.$$
\begin{enumerate}
    \item Déterminer le domaine de définition de $f$.
    
    \item Résoudre l'équation $y = f(x)$, c'est-à-dire déterminer $x$ en fonction de $y$.   
    Indication : exprimer $x$ sous la forme $x = \frac{ay+b}{cy+d}$.
    Quelle valeur $y_0$ de $y$ faut-il exclure ?
    
    \item On définit $g(y) = \frac{ay+b}{cy+d}$ (où $a,b,c,d$ ont été déterminés à la question précédente). Montrer que $g$ est la bijection réciproque de $f$, c'est-à-dire 
    $$(g \circ f)(x) = x  \qquad \text{ pour tout } x \neq 3$$
    et 
    $$(f \circ g)(y) = y  \qquad \text{ pour tout } y \neq y_0.$$
\end{enumerate} 
\finenonce

\indication
On calcule $x$ en fonction de $y$. En isolant $x$, on doit trouver : $x= \frac{3y-1}{y-2}$.
\finindication

\correction
\sauteligne
\begin{enumerate}
    \item Le domaine de définition de $f(x) = \frac{2x-1}{x-3}$ est $\Rr \setminus \{ 3 \}$.
    
    \item Fixons $y$ et résolvons l'équation $y = f(x)$, avec $x\neq3$, on a les équivalences :   
    \begin{align*}
        y = f(x)  
        & \iff y = \frac{2x-1}{x-3} \iff (x-3)y = 2x-1 \\
        & \iff xy -2x = 3y-1 \iff x(y-2) = 3y-1 \\
        & \iff x =  \frac{3y-1}{y-2} \quad \text{ et } y \neq 2 \\
     \end{align*} 
    Ces calculs sont valables pour $x \neq 3$ et $y \neq 2$.
    (On pourrait vérifier que l'équation $f(x) = 2$ n'a pas de solution.)
    
    
    \item 
    Définissons $g(y) =  \frac{3y-1}{y-2}$ comme fonction définie sur $\Rr \setminus \{ 2 \}$.
    Vérifions que $g$ est la bijection réciproque de $f$.
    
    D'une part, avec $x \neq 3$,  
    $$g \circ f(x) 
    = g \left( \frac{2x-1}{x-3} \right) 
    = \frac{3\frac{2x-1}{x-3}-1}{\frac{2x-1}{x-3}-2}
    = \frac{3(2x-1) - (x-3)}{2x-1 - 2(x-3)}
    = \frac{5x}{5}
    = x
    $$
    
    D'autre part, avec $y \neq 2$,
    $$f \circ g(y) 
    = f \left( \frac{3y-1}{y-2} \right) 
    = \frac{2\frac{3y-1}{y-2}-1}{\frac{3y-1}{y-2}-3} 
    = \frac{2(3y-1)-(y-2)}{3y-1 -3(y-2)}     
    = \frac{5y}{5}
    = y
    $$    
    
    Cela prouve que $f : \Rr \setminus \{ 3 \} \to \Rr \setminus \{ 2 \}$ et
    $g : \Rr \setminus \{ 2 \} \to \Rr \setminus \{ 3 \}$
    sont des bijections réciproques l'une de l'autre.
\end{enumerate} 
\fincorrection
\finexercice


\end{document}
