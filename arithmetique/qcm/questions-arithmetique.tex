%%%%%%%%%%%%%%%%%%%%%%%%%%%%%%%%%%%%%%%%%%%%%
\qcmtitle{Arithmétique}

\qcmauthor{Arnaud Bodin, Barnabé Croizat, Christine Sacré}



%%%%%%%%%%%%%%%%%%%%%%%%%%%%%%%%%%%%%%%%%%%%%
\section{Arithmétique}


%--------------------------------------------
\subsection{pgcd | Facile}


\begin{question}
On considère $a = 28$ et $b = 42$. Quelles sont les affirmations vraies ?
    \begin{answers} 
        \bad{Les diviseurs communs à $a$ et à $b$ sont : $1$, $2$, $7$.}
        \bad{$14$ est un diviseur de $a$ mais pas de $b$.}        
        \good{$6$ est un diviseur de $b$ mais pas de $a$.}
        \good{$84$ est un multiple de $a$ et de $b$.}
    \end{answers}
    \begin{explanations} 
     Les diviseurs communs à $a$ et à $b$ sont : $1$, $2$, $7$ et $14$.
     Le ppcm de $a$ et $b$ est $84$.   
    \end{explanations}
\end{question}


\begin{question}
    Quelles sont les valeurs qui correspondent à la division euclidienne $a=bq+r$ de $a$ par $b$ ?
    \begin{answers}
        \good{$a=48$, $b=7$, $q=6$, $r=6$}
        \good{$a=101$, $b=11$, $q=9$, $r=2$} 
        \bad{$a=56$, $b=9$, $q=5$, $r=11$}
        \bad{$a=123$, $b=10$, $q=13$, $r=-7$}        
    \end{answers}
    \begin{explanations} 
    $48 = 7 \times 6 + 6$ \\
    $101 = 11 \times 9 + 2$ \\
    $56 = 9 \times 6 + 2$. Attention $56 = 9 \times 5 + 11$, mais on n'a pas $0 \le 11 \le 9-1$, donc cette écriture n’est pas la division euclidienne de $56$ par $9$.\\  
    $123 = 10 \times 12 + 3$. Attention $123 = 10 \times 13 + (-7)$, mais on n'a pas $0 \le -7 \le 10-1$, donc cette écriture n’est pas la division euclidienne de $123$ par $10$.\\             
    \end{explanations}
\end{question}


\begin{question}
    Quelles sont les affirmations vraies ?
    \begin{answers} 
        \good{$456$ est divisible par $3$.}
        \bad{$754$ est divisible par $4$.}
        \bad{$5552$ est divisible par $5$.}
        \bad{$987$ est divisible par $9$.}
    \end{answers}
    \begin{explanations} 
     Critère de divisibilité par $3$ : la somme des chiffres est divisible par $3$. \\
     Critère de divisibilité par $2$ : le dernier chiffre est pair. \\    
     Pour décider si un entier est divisible par $4$ : diviser l'entier par $2$ et appliquer le critère de divisibilité par $2$. \\  
      Critère de divisibilité par $5$ : le dernier chiffre est $0$ ou $5$. \\ 
     Critère de divisibilité par $9$ : la somme des chiffres est divisible par $9$. \\                  
    \end{explanations}
\end{question}

\begin{question}
    Quel est le reste $r$ dans la division euclidienne de $145$ par $13$ ?
    \begin{answers} 
        \bad{$r=0$}
        \good{$r=2$}
        \bad{$r=7$}
        \bad{$r=-11$}
    \end{answers}
    \begin{explanations} 
     La division euclidienne de $145$ par $13$ nous donne l'écriture : $145 = 13 \times 11 + 2$. Le reste $r$ est donc $2$.\\
     Il est vrai que $145 = 13 \times 12 - 11$, mais cela ne correspond pas à une division euclidienne puisque le reste $r$ n'est pas compris entre $0$ et $13-1=12$.
    \end{explanations}
\end{question}

%--------------------------------------------
\subsection{pgcd | Moyen}


\begin{question}
    Soit $a=bq+r$ la division euclidienne de $a$ par $b$.
    Quelle condition définit le reste $r$ ?
    \begin{answers} 
        \bad{$0 \le r < a$}
        \good{$0 \le r < b$}
        \bad{$0 \le r \le q$}
        \bad{$0 \le r < q$}                
    \end{answers}
    \begin{explanations}
    Dans la division euclidienne, on a $0 \le r \le b-1$, c’est-à-dire $0 \le r < b$ puisque $r$ est un entier. 
    Cela permet d'avoir l'unicité du quotient $q$ et du reste $r$.
    \end{explanations}
\end{question}

\begin{question}
    Pour $a=220$ et $b=60$, quelles sont les affirmations vraies ?
    \begin{answers} 
        \bad{$\ppcm(a,b) = 440$.}
        \bad{$440$ est un multiple commun à $a$ et $b$.}
        \good{$10$ est un diviseur commun à $a$ et $b$.}
        \good{$\pgcd(a,b) = 20$.}
    \end{answers}
    \begin{explanations} 
        Le plus grand diviseur commun à $a=220$ et $b=60$ est $\pgcd(220,60) = 20$ (on peut l'obtenir via l'algorithme d'Euclide, on en dressant les listes exhaustives des diviseurs communs à $220$ et $60$). Puisque $10$ est un diviseur de $20$, $10$ est bien un diviseur commun à $a$ et $b$ (ce qui se voit sur l'écriture des deux nombres : ils finissent par $0$).\\
    En revanche $440$ n'est pas un multiple de $b=60$ (on a $60 \times 7 = 420$ et $60 \times 8 = 480$). On peut d'ailleurs calculer que $\ppcm(a,b) = \frac{a \times b}{\pgcd(a,b)} = 660$.
    \end{explanations}
\end{question}



\begin{question}
    Grâce à l'application de l'algorithme d'Euclide, on obtient pour $a=630$ et $b=165$ :
    \begin{answers} 
        \good{$\pgcd(a,b) = \pgcd(165,135)$}
        \good{$\pgcd(a,b) = \pgcd(135,30)$}
        \bad{$\pgcd(a,b) = \pgcd(30,0)$}
        \good{$\pgcd(a,b) = 15$}
    \end{answers}
    \begin{explanations} 
     L'algorithme d'Euclide nous donne :\\
     $$ 630 = 165 \times 3 + 135 $$
     $$ 165 = 135 \times 1 + 30 $$
     $$ 135 = 30 \times 4 + \boxed{ \; 15 \; } $$
     $$ 30 = 15 \times 2 + 0 $$
     Ainsi on a :
     $$ \pgcd(a,b) = \pgcd(165,135) = \pgcd(135,30) = \pgcd(30,15) = \pgcd(15,0) = 15 $$
     En revanche, $ \pgcd(30,0) = 30 \neq \pgcd(a,b)$.
    \end{explanations}
\end{question}


\begin{question}
    Soit $a>0$ un entier strictement positif dont le reste dans la division euclidienne par $8$ est $r=5$. Quelles sont les affirmations vraies ?
    \begin{answers} 
        \bad{$a$ est pair.}
        \good{$a$ est impair.}
        \bad{$a$ est nécessairement divisible par $13$.}
        \good{$(a-5)$ est un multiple de $8$.}
    \end{answers}
    \begin{explanations} 
     Puisque le reste dans la division euclidienne de $a$ par $8$ est $5$, on peut écrire $ a = 8k + 5 $, avec $k$ un nombre entier (positif car $a > 0$).\\
     On peut réécrire $a = 8k+5 = 2(4k+2) + 1$ : ainsi $a$ est impair.\\
     Pour $k=0$, on a $a=5$ qui n'est pas divisible par $13$ (ou aussi pour $k=2$ avec $a = 21$ par exemple).\\
     Puisqu'on a $(a-5) = 8k$, cela signifie que $(a-5)$ est bien un multiple de $8$.
    \end{explanations}
\end{question}


\begin{question}
    Pour $a=24$ et $b=8$, on a :
    \begin{answers} 
        \bad{$\ppcm(a,b) = 8$.}
        \good{$\ppcm(a,b) = 24$.}
        \good{$a$ est un multiple de $b$.}
        \bad{$a$ est dans la liste des diviseurs de $b$.}
    \end{answers}
    \begin{explanations}   
     $a$ étant un multiple de $b$ (on a $24 = 8 \times 3$), on a immédiatement $\pgcd(a,b) = b$ et $\ppcm(a,b) = a$.
    \end{explanations}
\end{question}



%--------------------------------------------
\subsection{pgcd | Difficile}

\begin{question}
    On considère $a,b$ et $d$ des entiers tels que $d | a$ et $d | b$.
    Quelles sont les affirmations vraies ?
    \begin{answers} 
        \good{$d | a+b$}        
        \good{$d | a-b$}
        \good{$d | a \times b$}
        \bad{$d | \frac{a}{b}$}
    \end{answers}
    \begin{explanations}
        L'affirmation $d | \frac{a}{b}$ est fausse et n'a même pas toujours de sens. Le reste est vrai.
    \end{explanations}
\end{question}


\begin{question}
    On considère $a,b$ et $n$ des entiers tels que $a | n$ et $b | n$.
    Quelles sont les affirmations vraies ?
    \begin{answers} 
        \bad{$a+b | n$}
        \bad{$a \times b | n$}
        \bad{$a+b | n^2$}
        \good{$a \times b | n^2$}
    \end{answers}
    \begin{explanations} 
    Si $a | n$ et $b | n$ alors $ab$ divise $n \times n  = n^2$. Les autres affirmations sont fausses. Trouver des contre-exemples, du style : $2|12$ et $3|12$ mais $2+3$ ne divise pas $12$.
    \end{explanations}
\end{question}


\begin{question}
    Soit $a_1$ un entier dont le reste dans la division euclidienne par $5$ est $r_1 = 2$. Soit $a_2$ un entier dont le reste dans la division euclidienne par $5$ est $r_2 = 3$. Quelles sont alors les affirmations vraies ?
    \begin{answers} 
        \good{Le reste de la division euclidienne de $a_1 + a_2$ par $5$ est $0$.}
        \bad{Le reste de la division euclidienne de $a_1 + a_2$ par $5$ est $5$.}
        \good{Le reste de la division euclidienne de $2a_1 + 2a_2$ par $5$ est $0$.}
        \good{L'écriture décimale de $2a_1 + 2a_2$ finit par le chiffre $0$.}
    \end{answers}
    \begin{explanations} 
    Les divisions euclidiennes par $5$ nous donnent : $a_1 = 5k_1 + 2$ et $a_2 = 5k_2 + 3$. On a ainsi :
    $$ a_1 + a_2 = 5(k_1+k_2) + 5 = 5(k_1+k_2+1) + 0 $$
    La dernière écriture correspond bien à la division euclidienne de $a_1 + a_2$ par $5$ car le reste (c'est $0$) est bien compris entre $0$ et $4$.\\
    De même, on calcule :
    $$ 2a_1 + 2a_2 = 2 \times 5(k_1+k_2+1) = 5 (2k_1 + 2k_2 + 2) = 10 (k_1 + k_2 + 1) $$
    Aussi $2a_1 + 2a_2$ est un entier divisible par $5$ et par $10$, donc son écriture décimale se termine par $0$.
    \end{explanations}
\end{question}



\begin{question}
    Soit $a>0$ un entier impair qui est un multiple de $3$. Quelles sont alors les affirmations vraies ?
    \begin{answers} 
        \bad{$a$ est un multiple de $6$.}
        \bad{L'écriture décimale de $a$ finit nécessairement soit par $7$ soit par $9$.}
        \good{$\pgcd(a,3) = 3$.}
        \good{$\ppcm(a,3) = a$.}
    \end{answers}
    \begin{explanations} 
    Les multiples positifs de $3$ s'écrivent $3N$, avec $N$ un entier positif. Si $N=2k$ est pair, alors $3N = 6k$ est un entier pair. Donc $a$, notre entier impair multiple de $3$, s'écrit $a = 3N$ avec $N = 2k+1$ un nombre impair ; ou encore $a = 3(2k+1) = 6k + 3$.\\
    Le reste de la division euclidienne de $a$ par $6$ est $3$. Donc $a$ n'est pas un multiple de $6$.\\
    Pour $k=2$ par exemple, on a $a = 6 \times 2 + 3 = 15$ qui est un entier impair, multiple de $3$, dont l'écriture décimale ne finit ni par $7$ ni par $9$.\\
    La liste des diviseurs de $3$ se réduit à $1$ et $3$. Puisque $3$ divise $a$, $3$ est un multiple commun à $3$ et $a$ : on a donc $\pgcd(a,3) = 3$. Par conséquent, on a aussi $\ppcm(a,3) = a$ de sorte que $\pgcd(a,3) \times \ppcm(a,3) = a \times 3$.
    \end{explanations}
\end{question}



\begin{question}
    Soient $a$ et $b$ deux entiers positifs tels que $\pgcd(a,b) = 10$ et $\ppcm(a,b) = 140$. Quelles sont les affirmations vraies ?
    \begin{answers} 
        \good{$\pgcd(2a,2b) = 20$}
        \bad{$\ppcm(2a,2b) = 70$}
        \bad{$\pgcd(2a,2b) = 10$}
        \good{$\ppcm(2a,2b) = 280$}
    \end{answers}
    \begin{explanations} 
    On utilise la relation $\pgcd(a,b) \times \ppcm(a,b) = a \times b$ pour obtenir $ab = 10 \times 140 = 1400$.\\
    On a alors $\pgcd(2a,2b) = 2 \times \pgcd(a,b) = 2 \times 10 = 20$.\\
    Mais on obtient aussi 
    $$ \pgcd(2a, 2b) \times \ppcm(2a,2b) = (2a) \times (2b) = 4 \times ab = 5600$$
    donc 
    $$\ppcm(2a,2b) = \frac{5600}{\pgcd(2a,2b)} = \frac{5600}{20} = 280 $$
    Remarquez qu'on a donc $\ppcm(2a,2b) = 2 \times \ppcm(2a,2b)$, et plus généralement $\ppcm(na,nb) = |n| \ppcm(a,b)$.
    \end{explanations}
\end{question}




%--------------------------------------------
\subsection{Théorème de Bézout | Facile}



\begin{question}
    Soient deux entiers $a,b$ tels que $\pgcd(a,b)=1$.
    Quelles sont les affirmations vraies ?
    \begin{answers} 
        \bad{$a$ et $b$ sont des nombres premiers.}
        \good{$a$ et $b$ sont des nombres premiers entre eux.}
        \good{Il existe $u,v\in\Zz$ tels que $au+bv=1$.}        
        \good{Il existe $u,v\in\Zz$ tels que $au+bv=2$.}
    \end{answers}
    \begin{explanations} 
      $\pgcd(a,b)=1$ est la définition de   $a$ et $b$ sont des nombres premiers entre eux.
      Le théorème de Bézout affirme qu'il existe $u,v\in\Zz$ tels que $au+bv=1$.
      En multipliant cette égalité par $2$, on obtient $a(2u)+b(2v)=2$.
    \end{explanations}
\end{question}


\begin{question}
    Soient $a,b,c$ des entiers tels que $a | bc$.
    Dans le lemme de Gauss, quelle est la condition pour pouvoir conclure que $a|c$ ?
    \begin{answers} 
        \good{$\pgcd(a,b)=1$}        
        \bad{$\pgcd(a,c)=1$}
        \bad{$\pgcd(b,c)=1$}
        \bad{$a$, $b$ et $c$ sont des nombres premiers.}
    \end{answers}
    \begin{explanations} 
    Lemme de Gauss : si $a | bc$ et $\pgcd(a,b)=1$ alors $a|c$.
    \end{explanations}
\end{question}


\begin{question}
 Soit $a$ et $b$ deux entiers tels que $\pgcd(a,b) = 4$. Alors on peut trouver deux entiers $u$ et $v$ tels que :
    \begin{answers} 
        \bad{$au-bv=2$}        
        \bad{$au+bv=2$}
        \good{$au-bv=4$}
        \good{$au+bv=12$}
    \end{answers}
    \begin{explanations} 
    Une égalité $au \pm bv = 2$ nous indiquerait que tout diviseur de $a$ et $b$ diviserait $2$ donc $\pgcd(a,b) = 1$ ou $2$, ce qui n'est pas le cas ici.\\
    Le théorème de Bézout nous garantit l'existence de deux entiers $U$ et $V$ tels que $aU + bV = 4$. Si l'on prend $v=-V$, on obtient $au-bv=4$. Si l'on multiplie l'égalité de Bézout par $3$, on a alors $a \times (3U) + b \times (3V) = 3 \times 4 = 12$.
    \end{explanations}
\end{question}



%--------------------------------------------
\subsection{Théorème de Bézout | Moyen}

\begin{question}
    Soient deux entiers positifs $a,b$, on calcule le pgcd de $a$ et $b$ par l'algorithme d'Euclide.
    La première étape est d'écrire la division euclidienne de $a$ par $b$ : $a=bq+r$.
    Quelle est la second étape ?
    \begin{answers} 
        \bad{La division de $a$ par $r$.}        
        \good{La division de $b$ par $r$.}        
        \bad{La division de $q$ par $r$.}
        \bad{Cela dépend des valeurs de $a$ et $b$.}
    \end{answers}
    \begin{explanations} 
        Une conséquence de l’égalité est que $\pgcd(a,b)=\pgcd(b,r)$. On remplace donc $a$ par $b$ et $b$ par $r$ dans l’étape suivante, c’est-à-dire qu’on fait la division euclidienne de $b$ par $r$.      
    \end{explanations}
\end{question}


\begin{question}
    Soient deux entiers positifs $a,b$ et $d=\pgcd(a,b)$.
    Quelles sont les affirmations vraies ?
    \begin{answers} 
        \bad{Il existe $u,v\in\Zz$ uniques tels que $au+bv=d$.}
        \good{Il existe $u,v\in\Zz$ tels que $au+bv=d$.}
        \bad{Il existe $u,v\in\Nn$ uniques tels que $au+bv=d$.}        
        \bad{Il existe $u,v\in\Nn$ tels que $au+bv=d$.}
    \end{answers}
    \begin{explanations} 
     Il existe $u,v\in\Zz$ tels que $au+bv=d$. $u$ et $v$ ne sont pas uniques. Comme $a,b,d>0$, $d \le a$ et $d \le b$ alors soit $u$ soit $v$ sera négatif.         
    \end{explanations}
\end{question}


\begin{question}
    Pour $a=453$ et $b=201$, l'algorithme d'Euclide (étendu) fournit des coefficients de Bézout $u$ et $v$ tels que $au+bv=\pgcd(a,b)$ avec :
    \begin{answers} 
        \bad{$u=4$, $v=-9$, $\pgcd(a,b)=1$.}        
        \bad{$u=-12$, $v=27$, $\pgcd(a,b) = 51$.}
        \bad{$u=1$, $v=-2$, $\pgcd(a,b)=51$}
        \good{$u=4$, $v=-9$, $\pgcd(a,b)=3$.}
    \end{answers}
    \begin{explanations} 
        L'algorithme d'Euclide nous fournit :
        $$ 453 = 2 \times 201 + 51$$
        $$201 = 3 \times 51 + 48 $$
        $$ 51 = 1 \times 48 + 3$$
        $$48 = 16 \times 3 + 0 $$
        Ainsi on a $\pgcd(a,b) = 3$. En remontant cet algorithme, on obtient :
        $$ 3 = 51 - 48 = 201 \times (-1) + 51 \times 4 = 453 \times 4 + 201 \times (-9)$$ 
    \end{explanations}
\end{question}


\begin{question}
    Pour les entiers $a,b$ suivants, les $u,v$ donnés sont-ils des coefficients de Bézout, c'est-à-dire tels que $au+bv=\pgcd(a,b)$ ?
    \begin{answers} 
        \bad{$a=7$, $b=11$, $u=2$, $v=-3$}   
        \bad{$a=20$, $b=55$, $u=6$, $v=-2$}
        \good{$a=28$, $b=12$, $u=1$, $v=-2$}
        \good{$a=36$, $b=15$, $u=-2$, $v=5$}        
    \end{answers}
    \begin{explanations}
    Pour $a=7$, $b=11$, $u=2$, $v=-3$, on a $\pgcd(a,b)=1$ et $au+bv= -19 \neq 1$. \\
    Pour $a=20$, $b=55$, $u=6$, $v=-2$, on a $\pgcd(a,b)=5$ et $au+bv= 10 \neq5$. \\
    Pour $a=28$, $b=12$, $u=1$, $v=-2$, on a $\pgcd(a,b)=4$ et $au+bv=4$. \\
    Pour $a=36$, $b=15$, $u=-2$, $v=5$, on a $\pgcd(a,b)=3$ et $au+bv=3$. \\  
%   On note $d=\pgcd(a,b)$. Les coefficients de Bézout vérifient $au+bv=d$. \\
%    $a=7$, $b=11$, $d=1$, $u=-3$, $v=2$ \\ 
%    $a=20$, $b=55$, $d=5$, $u=3$, $v=-1$ \\             
%    $a=28$, $b=12$, $d=4$, $u=1$, $v=-2$ \\    
%    $a=36$, $b=15$, $d=3$, $u=2$, $v=-5$ \\`
    \end{explanations}
\end{question}

\begin{question}
 Pour $a=41$ et $b=7$, on a notamment l'égalité $a \times (-3) + b \times 18 = 3$. Que peut-on en conclure ?
    \begin{answers} 
        \bad{$\pgcd(a,b)=3$.}        
        \good{$\pgcd(a,b)$ est un diviseur de $3$.}
        \good{Comme $3$ ne divise pas $7$ alors $a$ et $b$ sont premiers entre eux.}
        \bad{$-3$ et $18$ sont premiers entre eux.}
    \end{answers}
    \begin{explanations} 
    Comme $\pgcd(a,b)$ divise $a$ et $b$, il divise aussi $a \times (-3) + b \times 18 = 3$. Donc $\pgcd(a,b)$ est un diviseur de $3$ : c'est donc soit $3$, soit $1$. Mais puisque $3$ ne divise pas $b$ (ni $a$ d'ailleurs), on a donc $\pgcd(a,b)=1$ : $a$ et $b$ sont premiers entre eux.\\
    Enfin, les nombres $-3$ et $18$ sont divisibles par $3$.
    \end{explanations}
\end{question}


\begin{question}
 Soit deux nombres entiers $a$ et $b$ tels que $5a^2 - 4b^2 = 1$. Quelles sont les affirmations vraies ?
    \begin{answers} 
        \good{$\pgcd(a^2,b^2) = 1$.}        
        \good{$\pgcd(5a,4b) = 1$.}
        \bad{$5$ divise $4b^2$.}
        \good{$4$ divise $5a^2-1$.}
    \end{answers}
    \begin{explanations} 
    L'égalité fournie peut s'écrire sous les formes $5 \times a^2 + (-4) \times b^2 = 1 = a \times (5a) + (-b) \times 4b = 1$. Ce sont notamment des identités de Bézout pour les couples $(a^2,b^2)$ et $(5a,4b)$ qui sont donc premiers entre eux.\\
    Si $5$ était un diviseur de $4b^2$, il diviserait $(5a^2 - 4b^2)$ et donc $1$ ce qui est impossible.\\
    Enfin ayant $5a^2 - 1 = 4 b^2$, le nombre $5a^2-1$ est bien un multiple de $4$.
    \end{explanations}
\end{question}


%--------------------------------------------
\subsection{Théorème de Bézout | Difficile}


\begin{question}
    Quelles sont les affirmations vraies concernant l'algorithme d'Euclide ?
    \begin{answers} 
        \bad{Il se peut que le processus n'aboutisse pas à cause d'un nombre infini de divisions à effectuer.}        
        \bad{Il se peut que le processus ne fournisse pas le pgcd correct.}
        \good{Le pgcd est le dernier reste non nul.}
        \good{L'algorithme étendu permet en plus de calculer des coefficients de Bézout.}
    \end{answers}
    \begin{explanations}
        L'algorithme d'Euclide fournit un résultat \emph{correct} en un nombre \emph{fini} d'étapes. La remontée de l'algorithme d'Euclide permet de calculer des coefficients de Bézout.
    \end{explanations}
\end{question}


\begin{question}
 Soit $n$ un entier tel que $5n$ soit un multiple de $7$. Quelles sont alors les affirmations vraies ?
    \begin{answers} 
        \good{$n$ est un multiple de $7$.}        
        \bad{$5$ divise $7n$.}
        \good{$7$ divise $n$.}
        \bad{$35$ divise $n$.}
    \end{answers}
    \begin{explanations} 
    D'après le lemme de Gauss, puisque $7 | 5n$ et que $\pgcd(5,7) = 1$, on a $ 7 | n $ : ceci revient à dire que $n$ est un multiple de $7$.\\
    En revanche si l'on prend $n=7$, on constate que $5n = 35$ est bien multiple de $7$ mais que $5$ ne divise pas $7n=49$ et que $35$ ne divise pas $n=7$.
    \end{explanations}
\end{question}


\begin{question}
 Soient $5$ entiers relatifs $a,b,c,u,v$ tels que $au+bv=1$ et $a | bc$. Quelles sont alors les affirmations vraies ?
    \begin{answers} 
        \bad{$\pgcd(a,c)=1$.}        
        \good{$\pgcd(a,b)=1$.}
        \good{$a|c$.}
        \good{$\pgcd(a,c) = |a|$.}
    \end{answers}
    \begin{explanations} 
    $au+bv=1$ est une identité de Bézout qui garantit que $\pgcd(a,b)=1$. D'après le lemme de Gauss, puisque $a|bc$, on a alors $a|c$.\\
    Puisque $a|c$, $|a|$ est un diviseur de $c$. Or c'est le plus grand diviseur de $a$ : donc $|a| = \pgcd(a,c)$.\\
    Un contre-exemple pour établir que $\pgcd(a,c)$ n'est pas nécessairement égal à $1$ peut par exemple être $a=5$, $b=7$ (bien premiers entre eux) et $c=10$. On a bien $a | bc $ mais $\pgcd(a,c) = 5$. Plus généralement, on peut toujours respecter la condition $a | bc$ avec $c=a$, ce qui contredit $\pgcd(a,c)=1$ dès que $a$ n'est pas égal à $\pm 1$.
    \end{explanations}
\end{question}



%--------------------------------------------
\subsection{Nombres premiers | Facile}


\begin{question}
    Les entiers suivants sont-ils des nombres premiers ?
    \begin{answers} 
        \good{$107$}
        \good{$113$}        
        \bad{$145$}
        \bad{$153$}        
    \end{answers}
    \begin{explanations} 
        $107$ et $113$ sont des nombres premiers ;
        $145 = 5 \times 29$ ;  $153 = 3^2 \times 17$.
    \end{explanations}
\end{question}


\begin{question}
    Quelles sont les affirmations vraies ?
    \begin{answers} 
        \bad{Tout nombre impair supérieur à $3$ est premier.}
        \good{Tout nombre premier supérieur à $3$ est impair.}
        \good{Il existe une infinité de nombres premiers impairs.}        
        \bad{Il existe une infinité de nombres premiers pairs.}
    \end{answers}
    \begin{explanations}
    Par exemple $9$ est un nombre impair qui n'est pas premier. 
    Le seul nombre premier pair est $2$, tous les autres sont impairs et il y en a une infinité.     
    \end{explanations}
\end{question}


\begin{question}
 Les entiers suivants sont-ils des nombres premiers ?
    \begin{answers} 
        \bad{$161$}        
        \bad{$169$}         
        \bad{$171$}               
        \good{$179$}
    \end{answers}
    \begin{explanations} 
    On a $161 = 7 \times 23$, $169 = 13^2$ et $171$ est divisible par $9$ (car la somme de ses chiffres fait $9$).\\
    En revanche, $179$ n'est pas divisible par $2,3, 5, 7, 11, 13$ ce qui garantit sa primalité. 
    \end{explanations}
\end{question}


%--------------------------------------------
\subsection{Nombres premiers | Moyen}


\begin{question}
    Quelles sont les affirmations vraies ?
    \begin{answers} 
        \good{La somme de deux nombres premiers $\ge3$ n'est jamais un nombre premier.}
        \good{Le produit de deux nombres premiers $\ge3$ n'est jamais un nombre premier.}
        \bad{Il existe un nombre premier $p\ge3$ tel que $p+1$ soit aussi premier.}
        \good{Il existe un nombre premier $p\ge3$ tel que $p+2$ soit aussi premier.}
    \end{answers}
    \begin{explanations}
        Le produit de deux nombres premiers n'est jamais un nombre premier (par définition de ce qu'est un nombre premier).
        Pour la somme, cela peut arriver, par exemple $2+3=5$, mais pour deux nombres premiers $\ge3$, ils sont impairs, donc la somme est paire et n'est pas un nombre premier.
        De même si $p\ge3$ est premier, il est impair, donc $p+1$ est pair et n'est pas premier.
        Par contre pour $p=11$ alors $p+2=13$ est aussi premier, d'autres exemples sont $17$ et $19$ ou bien $101$ et $103$.
    \end{explanations}
\end{question}

\begin{question}   
    Soient $p$ un nombre premier et $a,b$ des entiers avec $p | ab$.
    Par application du lemme d'Euclide, quelles sont les affirmations vraies ?
    \begin{answers} 
        \bad{$p$ divise $a$ et $p$ divise $b$.}
        \good{$p$ divise $a$ ou $p$ divise $b$.}
        \bad{$p$ divise $a$ ou $p$ divise $b$, mais pas les deux en même temps.}
        \bad{$p$ ne divise ni $a$, ni $b$.}
    \end{answers}
    \begin{explanations} 
        Lemme d'Euclide : Si $p$ premier et $p | ab$, alors $p|a$ ou $p|b$.
        Les autres affirmations sont fausses. Voici des contre-exemples :
        $2 | (3 \times 4)$ mais $2$ ne divise pas $3$, mais divise bien $4$ ; 
        $2 | (4 \times 6)$ et $2|4$ et $2|6$.        
    \end{explanations}
\end{question}


\begin{question}
    Soit $n$ un entier tel que $n^2-1$ est un multiple de $11$. Quelles sont les affirmations vraies ?
    \begin{answers} 
        \bad{$11$ divise $n-1$.}        
        \bad{$11$ divise $n+1$.}
        \good{($11$ divise $n-1$) ou ($11$ divise $n+1$).}
        \bad{($11$ divise $n-1$) et ($11$ divise $n+1$).}
    \end{answers}
    \begin{explanations} 
    $11$ est premier et divise $n^2-1 = (n-1)(n+1)$. D'après le lemme d'Euclide, soit $11 | (n-1)$, soit $11 | (n+1)$.\\
    Pour $n=10$, on a $11 | (10^2-1)$ mais $11$ ne divise pas $10-1=9$.\\
    Pour $n=12$, on a $11 | 12^2 - 1$ mais $11$ ne divise pas $12+1=13$.\\
    Et si $11$ divisait $n-1$ et $n+1$ alors $11$ diviserait  $n+1-(n-1) = 2$.
    \end{explanations}
\end{question}


\begin{question}
    À l'aide d'une calculatrice, quelle est l'écriture de la décomposition en produit de facteurs premiers de $N = 111 \, 111$ ?
    \begin{answers} 
        \bad{$N = 11 \times 10\,101$.}        
        \bad{$N = 3 \times 11 \times 3367$.}
        \bad{$N = 7 \times 33 \times 481$.}
        \good{$N = 3 \times 7 \times 11 \times 13 \times3713$.}
    \end{answers}
    \begin{explanations} 
    Les entiers $10\,101$, $3367$ et $481$ sont des multiples de $13$ !
    \end{explanations}
\end{question}


\begin{question}
 Soit $p \geq 3$ un nombre premier et $ p = 4q + r$ le résultat de sa division euclidienne par $4$. On peut alors avoir :
    \begin{answers} 
        \bad{$r=0$}        
        \good{$r=1$}
        \bad{$r=2$}
        \good{$r=3$}
    \end{answers}
    \begin{explanations} 
    Un nombre premier $\geq 3$ est nécessairement impair. Ceci exclut donc les possibilité $r=0$ et $r=2$ qui correspondent à des nombres pairs.\\
    On peut à titre d'exemple obtenir pour $p=3$ que $r=3$ ; et pour $p=5$ que  $r=1$.
    \end{explanations}
\end{question}



\begin{question}
 Soit $p$ un nombre premier tel que $10 < p < 100$. On note $A$ le chiffre des dizaines et $B$ le chiffre des unités de l'écriture décimale de $p$. Quelles sont les affirmations vraies ?
    \begin{answers} 
        \good{$A$ peut être pair.}        
        \bad{$B$ peut être pair.}
        \good{On peut avoir $A = B$.}
        \bad{On peut avoir $B = 9 - A$.}
    \end{answers}
    \begin{explanations}
    Pour $p = 23$ qui est premier, on a bien $A = 2$ qui est pair.\\ 
    Si le chiffre des unités $B$ est pair, alors $p$ est pair ce qui est impossible pour un nombre premier $\geq 3$.\\    
    Pour $p=11$ premier, on a bien $A=B$ (les autres nombres avec deux chiffres identiques sont justement les multiples de $11$, et ne sont donc pas premiers).\\    
    Si $B = 9-A$, alors la somme des chiffres de $p$ vaut $A + B = 9$ : ainsi $p$ est divisible par $9$, ce qui contredit sa primalité.
    \end{explanations}
\end{question}


%--------------------------------------------
\subsection{Nombres premiers | Difficile}


\begin{question}
    Les entiers suivants ont été factorisés correctement. 
    Quelles sont les écritures qui sont des décompositions en facteurs premiers ?
    \begin{answers} 
        \bad{$3\,025 = 1^3 \times 5^2 \times 11^2$}
        \bad{$1\,836 = 2^2 \times 3 \times 3^2 \times 17$}
        \bad{$1\,444\,716 = 2^2 \times 7^3 \times 9^2 \times13$}
        \good{$13\,915 = 5 \times 11^2 \times 23$}
    \end{answers}
    \begin{explanations} 
    Chaque facteur doit être de la forme $p_i^{\alpha_i}$ avec $p_i$ un nombre premier (donc pas $1$ ni $9$) et $\alpha_i>0$. En plus les $p_i$ doivent être deux à deux distincts. Avec ces contraintes la décomposition est unique (à l'ordre des facteurs près).
    \end{explanations}
\end{question}


\begin{question}
    Soient $a = 5^3 \times 11^2 \times 13^5 \times 19$ 
    et $b = 5^5 \times 7^4 \times 11 \times 19$  
    Quelles sont les affirmations vraies ?
    \begin{answers} 
        \bad{$\pgcd(a,b) = 5^3 \times 7^4 \times 11 \times 13^5 \times 19$}
        \bad{$\pgcd(a,b) = 5 \times 11 \times 19$}
        \good{$\ppcm(a,b) = 5^5 \times 7^4 \times 11^2 \times 13^5 \times 19$}        
        \bad{$\ppcm(a,b) = 5^5 \times 11^2 \times 19$}
    \end{answers}
    \begin{explanations}
    Pour le pgcd, on garde le plus petit exposant des décompositions de $a$ et $b$ ; pour le ppcm, on garde le plus grand exposant. \\
     $\pgcd(a,b) = 5^3 \times 11 \times 19$ \\         
     $\ppcm(a,b) = 5^5 \times 7^4 \times 11^2 \times 13^5 \times 19$   
    \end{explanations}
\end{question}


\begin{question}
 Soit $a = 79 \, 475 = 5^2 \times 11 \times 17^2$. Quelles sont les affirmations vraies ?
    \begin{answers} 
       \bad{$\pgcd(a,75) = 3 \times 5^2$}  
       \good{$\pgcd(a,75) = 5^2$}             
       \bad{$\ppcm(a,75) = 3 \times 11 \times 17^2$}         
       \bad{$75 | a$}
    \end{answers}
    \begin{explanations} 
    On a $75 = 3 \times 5^2$. En utilisant les décompositions en produits de facteurs premiers, on obtient :
    $$ \pgcd(a,75) = 5^2 = 25 \qquad ; \qquad \ppcm(a,75) = 3 \times 5^2 \times 11 \times 17^2$$
    Enfin $a$ n'est pas divisible par $3$ donc il n'est pas divisible par $75 = 3 \times 25$.
    \end{explanations}
\end{question}



\begin{question}
 Soit $p \geq 5$ un nombre premier et $N = (p+3)^2 - p^2$. Quelles sont les affirmations vraies ?
    \begin{answers} 
        \bad{$2 | N$.}        
        \good{$3 | N$.}
        \bad{$6 | N$.}
        \good{$p$ ne divise pas $N$.}
    \end{answers}
    \begin{explanations} 
    En développant, on constate que $N = 6p+9 = 6(p+1)+3 =  3(2p+3)$. $N$ est donc un multiple de $3$, non divisible par $6$ (le reste dans la division euclidienne par $6$ est $3$). $N$ est impair (produit de deux nombres impairs) et n'est donc pas divisible par $2$.\\
    Enfin, si $p|N$, on a $p | (6p+9)$ et donc $p|9$. Puisque $p$ est premier, cela signifie que $p=3$ ce qui est impossible car $p \geq 5$.
    \end{explanations}
\end{question}


%--------------------------------------------
\subsection{Congruences | Facile}

\begin{question}
    Quelles sont les affirmations vraies ?
    \begin{answers} 
        \bad{$31 \equiv 6 \; [12]$}
        \good{$42 \equiv 16 \; [13]$}
        \bad{$25 \equiv -11 \; [14]$}
        \good{$158 \equiv 8 \; [15]$}
    \end{answers}
    \begin{explanations} 
    $12$ ne divise pas $31-6 =25$ ; en fait $31 \equiv 7 \; [12]$. \\
    $13$ divise $42-16=26$ ; en fait $42 \equiv 16 \equiv 3 \; [13]$. \\
    $14$ ne divise pas $25-(-11)=36$ ; en fait $25 \equiv +11  \equiv -3 \; [14]$. \\
    $15$ divise $158-8=150$ ;  en fait $158 = 15\times10+8 \equiv 8 \; [15]$.
    \end{explanations}
\end{question}


\begin{question}
    Quelles sont les affirmations vraies ?
    \begin{answers}
    \bad{$456\,789 \equiv 0 \; [2]$}
    \good{$43\,210 \equiv 0 \; [5]$}        
    \bad{$23\,769 \equiv 3 \; [9]$}         
    \bad{$10\,326 \equiv 8 \; [10]$}
    \end{answers}
    \begin{explanations} 
    $456\,789$ est impair, donc n’est pas congru à $0$ modulo $2$. \\
    $43\,210$ est divisible par $5$ donc congru à $0$ modulo $5$. \\
    $23\,769$ est divisible par $9$ (la somme des chiffres est divisible par $9$) donc congru à $0$ modulo $9$. \\
    $10\,326 \equiv 6 \; [10]$, réduire modulo $10$ c'est garder le chiffre des unités.
    \end{explanations}
\end{question}


\begin{question}
 Si $x \equiv 2 \; [5]$, alors on a :
    \begin{answers} 
        \good{$x^2 \equiv 2x \; [5]$}        
        \bad{$3x \equiv -1 \; [5]$}
        \good{$x+1 \equiv 3 \; [5]$}
        \bad{$10x \equiv 2 \; [5]$}
    \end{answers}
    \begin{explanations} 
    D'après les propriétés arithmétiques des congruences et notre congruence initiale $x \equiv 2 \;[5]$ :\\
    en ajoutant $1$ : $x+1 \equiv 2+1 = 3 \; [5]$,\\
    en multipliant par $3$ : $3x \equiv 3 \times 2 = 6 \equiv 1 \;[5]$,\\
    en multipliant par $10$ : $10x \equiv 10 \times 2 = 20 \equiv 0 \;[5]$,\\
    Enfin on calcule : $x^2 \equiv 2 \times 2 \equiv 2x \;[5]$.
    \end{explanations}
\end{question}


\begin{question}
    Parmi les nombres $n$ ci-dessous, lequel vérifie à la fois $n \equiv 5 \;[14]$ et $n \equiv 1 \;[8]$ ?
    \begin{answers} 
        \bad{$n=47$}        
        \bad{$n=57$}
        \good{$n=89$}
        \bad{$n=103$}
    \end{answers}
    \begin{explanations} 
        On a bien $89 \equiv 5 \;[14]$ ($89 = 14 \times 6 + 5$) et $89 \equiv 1 \; [8]$ ($89 = 8 \times 11 + 1$).\\
        On calcule que $47 \equiv 7 \;[8]$, $57 \equiv 1 \; [14]$ et $103 \equiv 7 \;[8]$.
    \end{explanations}
\end{question}



%--------------------------------------------
\subsection{Congruences | Moyen}


\begin{question}
     Soient $a\equiv 2 \;[13]$ et $b \equiv 7 \; [13]$.
    Quelles sont les affirmations vraies ?
    \begin{answers}
        \good{$a+b \equiv 9 \;[13]$}        
        \good{$ab \equiv 1 \; [13]$}
        \good{$a^2 \equiv -9 \;[13]$}        
        \good{$b^3 \equiv 5 \;[13]$} 
    \end{answers}
    \begin{explanations} 
        Tout est vrai ! Modulo $13$, on a bien : 
        $$a+b = 2+7 = 9 \equiv 9,$$
        $$ab = 2 \times 7 = 14 \equiv 1,$$
        $$a^2 = 2^2 = 4 \equiv -9,$$
        $$b^3 = 7^3 = 343 \equiv 5.$$
    \end{explanations}
\end{question}

\begin{question}
    Soient $a\equiv b \;[n]$ et $c \equiv d \; [n]$.
    Quelles sont les affirmations vraies ?
    \begin{answers} 
        \bad{$a+b \equiv c+d \;[n]$}
        \good{$a+c \equiv b+d \;[n]$}
        \good{$a^2 \equiv b^2 \;[n]$}
        \good{$c^2 \equiv d^2 \;[n]$}
    \end{answers}
    \begin{explanations} 
     $a+c \equiv b+d \;[n]$ et  $a^k \equiv b^k \;[n]$ et aussi $c^k \equiv d^k \;[n]$.
    \end{explanations}

\end{question}


\begin{question}
 Soit $n$ un entier premier avec $3$. On peut alors affirmer :
    \begin{answers} 
        \bad{$ 2n \equiv 1 \;[3]$}        
        \bad{$2n \equiv -1\; [3]$}
        \good{$n^2 \equiv 1 \;[3]$}
        \bad{$n^2 \equiv -1 \;[3]$}
    \end{answers}
    \begin{explanations} 
    Puisque $n$ n'est pas un multiple de $3$, on a soit $n \equiv 1 \;[3]$ (cas 1) soit $n \equiv 2 \equiv -1 \;[3]$ (cas 2). Dans le cas 1, on a $2n \equiv 2 \equiv -1 \;[3]$, et dans le cas 2 on a $2n \equiv -2 \equiv 1 \; [3]$.\\
    Dans les deux cas, on aura $n^2 \equiv 1 \; [3]$.
    \end{explanations}
\end{question}


\begin{question}
    Soit $k$ un entier et $N = 5k^2-10k+4$. On peut affirmer :
    \begin{answers} 
        \good{$N \equiv 4 \;[5]$}        
        \bad{$N \equiv 5 \; [5]$}
        \good{$N \equiv 5k^2 \;[2]$}
        \bad{$N \equiv 1 \;[2]$}
    \end{answers}
    \begin{explanations} 
        Puisque $5 \equiv 10 \equiv 0 \;[5]$, on a $5k^2-10k \equiv 0 \;[5]$. Donc $N \equiv 4 \;[5]$.\\
        D'autre part, puisque $10 \equiv 4 \equiv 0 \;[2]$, on a $-10k+4 \equiv 0 \;[2]$. Donc $N \equiv 5k^2 \;[2]$. Le cas $k=2$ (ou tout autre entier pair) montre que l'on peut avoir $N \equiv 5k^2 \equiv 0 \;[2]$.
    \end{explanations}
\end{question}



%--------------------------------------------
\subsection{Congruences | Difficile}

\begin{question}
    Soit $p$ un nombre premier et $x$ un entier.
    Quel(s) énoncé(s) du petit théorème de Fermat sont corrects ?
    \begin{answers} 
        \bad{$x^p \equiv p \;[x]$}
        \good{$x^p \equiv x \;[p]$}
        \bad{Si $p$ ne divise pas $x$, alors $x^{p-1} \equiv 0 \;[x]$}
        \bad{Si $p$ ne divise pas $x$, alors $x^{p-1} \equiv 0 \;[p]$}
    \end{answers}
    \begin{explanations} 
    Le théorème de Fermat stipule que $x^p \equiv x \;[p]$, et que si $p$ ne divise pas $x$ alors $x^{p-1} \equiv 1 \;[p]$.
    \end{explanations}
\end{question}


\begin{question}
    Quelles sont les affirmations vraies ?
    \begin{answers} 
        \bad{$2^8 \equiv 2 \;[8]$}        
        \bad{$3^{12} \equiv 3 \;[13]$}
        \bad{$18^7 \equiv 1 \; [19]$}
        \good{$4^{16} \equiv 1 \; [17]$}
    \end{answers}
    \begin{explanations}
    $8$ n'est pas un nombre premier, le petit théorème de Fermat ne s'applique pas. En fait $2^8 \equiv 0 \;[8]$ car $2^3 = 8 \equiv 0 \;[8]$. \\ 
    Petit théorème de Fermat, avec $p=13$,  $3^{12} \equiv 1 \;[13]$. \\
    $18^7 \equiv (-1)^7 \equiv -1  \equiv 18 \; [19]$ (le calcul n'a rien à voir avec le petit théorème de Fermat). \\
    Petit théorème de Fermat, avec $p=17$,  $4^{16} \equiv 1 \; [17]$.
    \end{explanations}
\end{question}


\begin{question}
 Soit un entier $k$ tel que $k \equiv 2 \;[7]$. Quelles sont les affirmations vraies ?
    \begin{answers} 
        \bad{$2k^2 + k \equiv k^3 \;[7]$}        
        \good{$3(k^4-k) \equiv 0 \;[7]$}
        \good{$14k-2 \equiv 5 \;[7]$}
        \bad{$k^{18} + k^{12} + k^6 \equiv k \;[7]$}
    \end{answers}
    \begin{explanations} 
    On calcule que $k^2 \equiv 4 \;[7]$ ; $k^3 \equiv 2^3 \equiv 1 \;[7]$ et
    $k^4 \equiv 2^4 \equiv 2 \;[7]$. On a alors :\\
    $2k^2 + k \equiv 2 \times 2^2 + 2 \equiv 10 \equiv 3 \;[7]$.\\
    $k^4 \equiv k \;[7]$ donc $3(k^4-k) \equiv 3 \times 0 \equiv 0 \;[7]$.\\
    $14k \equiv 7 \times 2k \equiv 0 \;[7]$ donc $14k-2 \equiv -2 \equiv 5 \;[7]$.\\
    Enfin on a $k^{18} \equiv k^{12} \equiv k^6 \equiv 1 \;[7]$ (c'est le théorème de Fermat, ou une conséquence directe de $k^3 \equiv 1 \;[7]$). Donc 
    $k^{18} + k^{12} + k^6 \equiv 3 \;[7]$.
    \end{explanations}
\end{question}


\begin{question}
 Pour quel(s) entier(s) $n$ a-t-on $10^{10} \equiv 7^{18} \; [n]$ ?
    \begin{answers} 
        \good{$n=3$}        
        \bad{$n=5$}
        \bad{$n=7$}
        \good{$n=9$}
    \end{answers}
    \begin{explanations} 
    On a $10 \equiv 7 \equiv 1 \;[3]$. Donc $10^{10} \equiv 1 \equiv 7^{18} \;[3]$.\\
    $10 \equiv 0 \;[5]$ donc $10^{10} \equiv 0 \;[5]$. Mais $7^{18} \equiv (7^4)^4 \times 7^2 \equiv 1^4 \times 49 \equiv -1 \;[5]$.\\
    $7^{18} \equiv 0 \;[7]$ mais $10^{10} \equiv 3^{10} \equiv 3^6 \times 3^4 \equiv 1 \times 81 \equiv 4 \;[7]$.\\
    Enfin, on a $10^{10} \equiv 1^{10} \equiv 1 \;[9]$, et également $7^{18} \equiv (-2)^{18} \equiv 2^{18} \equiv 8^6 \equiv (-1)^6 \equiv 1 \;[9]$.
    \end{explanations}
\end{question}


\begin{question}
 Quel est le chiffre des unités de $7^{100}$ ?
    \begin{answers} 
        \good{$1$}        
        \bad{$3$}
        \bad{$5$}
        \bad{$9$}
    \end{answers}
    \begin{explanations} 
    Le chiffre des unités est donné par la congruence modulo $10$.\\
    Puisque $7^2 = 49 \equiv (-1) \;[10]$, on a :
    $$ 7^{100} = (7^2)^{50} \equiv (-1)^{50} \equiv 1 \;[10]$$
    \end{explanations}
\end{question}

