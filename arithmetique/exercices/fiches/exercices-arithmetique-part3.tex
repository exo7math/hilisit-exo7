\documentclass[11pt,class=report,crop=false]{standalone}
\usepackage{exo7hilisit}

\newcommand{\pgcd}{\mathop{\mathrm{pgcd}}\nolimits} 
\newcommand{\ppcm}{\mathop{\mathrm{ppcm}}\nolimits}

\begin{document}
    


\entete{Hilisit}{Capacité mathématiques}

\titre{Arithmétique -- Partie 3 : Nombres premiers}

\bigskip
\bigskip



%%%%%%%%%%%%%%%%%%%%%%%%%%%%%%%%%%%%%%%%%%%%%%%%%%%%%%%%%%%%
%\section{Nombres premiers}

\exercice{}
\enonce
Trouver tous les nombres premiers plus petits que $100$.
\finenonce


\finexercice



\exercice{}
\enonce
Calculer la décomposition en facteurs premiers de $a$ puis de $b$, en déduire $\pgcd(a,b)$ et $\ppcm(a,b)$.
\begin{enumerate}
    \item $a=1500$, $b=1470$.
    \item $a=18\,135$, $b=92\,950$.
\end{enumerate}
\finenonce


\finexercice



\exercice{}
\enonce
Soit $p$ un nombre premier. Montrer que pour tout entier $k$ tel que $1 \leq k \leq p-1$, alors $p$ divise $\binom{p}{k}$.

\textit{On rappelle l'expression du coefficient binomial :}
$$ \binom{n}{k} = \frac{n!}{k!(n-k)!}$$
\finenonce


\finexercice


\end{document}


