\documentclass[11pt,a4paper]{report}
\usepackage{exo7hilisit}

\begin{document}

%%%%%%%%%%%%%%%%%%%%%%%%%%%%%%%%%%%%%%%%%%%%%%%%%%%%%%%%%%%%%%%%%%%%%%
%%%%%%%%%%%%%%%%%%%%%%%%%%%%%%%%%%%%%%%%%%%%%%%%%%%%%%%%%%%%%%%%%%%%%%

\entete{Hilisit}{Capacité mathématiques}

\titre{Cours -- \'Equations différentielles}

\bigskip
\bigskip

\begin{quote}
\center
\emph{
Les équations différentielles jouent un rôle important en mathématique mais s'appliquent aussi aux autres sciences. Elles apparaissent naturellement en mécanique (par exemple comme équations issues du principe fondamental de la mécanique), en électricité ou pour décrire la désintégration des éléments radioactifs. En biologie, elles permettent de décrire l'évolution des populations (d'animaux, de bactéries...) ou des concentrations de molécules.
L'objectif de ce chapitre est double : comprendre ce nouveau type d'équation et savoir résoudre des équations différentielles simples.
}
\end{quote}

\bigskip
\bigskip


\textbf{Sections}
\begin{enumerate}[label=\arabic*.]
  \item \textbf{Primitives}
  
  Thèmes : Rappels sur les primitives qui jouent un rôle important pour la résolution des équations différentielles.
  
  Objectifs : 
  Connaître la définition d'une primitive. 
  Connaître le lien entre deux primitives d'une même fonction. 
  Connaître les formules des primitives usuelles. 
  Savoir déterminer une primitive.
  
  
  \item \textbf{Notion d'équation différentielle}
  
  Thèmes : Introduction et motivation aux équations différentielles.

  Objectifs : 
  Comprendre ce qu'est une équation différentielle. 
  Savoir vérifier qu'une fonction est solution d'une équation différentielle.
  
  \item \textbf{\'Equations différentielles $y'=ay$}
    
  Thèmes : Résolution des équations différentielles $y'=ay$.
  
  Objectifs : 
  Connaître la formule de la solution d'une équation différentielle $y'=ay$.
  Comprendre ce qu'est une condition initiale d'une équation différentielle.
  Comprendre qu'il y a unicité d'une solution lorsqu'on impose une condition initiale.
  Savoir résoudre une équation différentielle $y'=ay$.
  Savoir trouver la solution vérifiant une condition initiale donnée.
  
  \item \textbf{\'Equations différentielles $y'=ay+b$ et $y'=ay+f$}
    
  Thèmes : Résolution des équations différentielles $y'=ay+b$ et $y'=ay+f$
  
  Objectifs : 
  Connaître le vocabulaire : second membre, équation homogène, solution particulière.
  Comprendre la structure d'une solution générale : "solutions de l'équation homogène + solution particulière".
  Savoir trouver une solution particulière à l'aide d'indications.
  Savoir trouver toutes les solutions à partir d'une solution particulière et des solutions de l'équation homogène.
  
\end{enumerate}


\bigskip
\bigskip

\begin{quote}
\center
\emph{Dans ce chapitre on se permettra de noter la fonction $x \mapsto f(x)$ simplement par $f(x)$ pour alléger l'écriture. Par exemple on dira qu'une primitive de $\cos(x)$ est $\sin(x)$ ou bien que $e^x$ est solution de l'équation différentielle $y'=y$.}
\end{quote}

\vfill

\begin{center}
\begin{minipage}{0.8\textwidth}
\center
Auteurs : Arnaud Bodin, Barnabé Croizat et Christine Sacré de l'université de Lille.
Certaines parties sont tirées d'un travail avec Cécile Mammez. Relecture de Pascal Romon.

  \medskip
  
Ce travail a été effectué en 2021-2022 dans le cadre d'un projet Hilisit porté  Unisciel.
\end{minipage}

  \medskip

\raisebox{1ex}{\includegraphics[height=1.8cm]{logo-unisciel}}\qquad\qquad
\includegraphics[height=2.2cm]{logo-ulille}

  \medskip
  
Ce document est diffusé sous la licence \emph{Creative Commons -- BY-NC-SA -- 4.0 FR}.


Sur le site Exo7 vous pouvez récupérer les fichiers sources.

\vspace*{0cm}

\end{center}


\newpage


%%%%%%%%%%%%%%%%%%%%%%%%%%%%%%%%%%%%%%%%%%%%%%%%%%%%%%%%%%%%%%%%%%%%%%
%%%%%%%%%%%%%%%%%%%%%%%%%%%%%%%%%%%%%%%%%%%%%%%%%%%%%%%%%%%%%%%%%%%%%%

\import{./}{equadiff-part1.tex}
\newpage

\import{./}{equadiff-part2.tex}
\newpage

\import{./}{equadiff-part3.tex}
\newpage

\import{./}{equadiff-part4.tex}
\newpage


%%%%%%%%%%%%%%%%%%%%%%%%%%%%%%%%%%%%%%%%%%%%%%%%%%%%%%%%%%%%%%%%%%%%%%
%%%%%%%%%%%%%%%%%%%%%%%%%%%%%%%%%%%%%%%%%%%%%%%%%%%%%%%%%%%%%%%%%%%%%%

\end{document}
