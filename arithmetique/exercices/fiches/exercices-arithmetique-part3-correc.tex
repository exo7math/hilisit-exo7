\documentclass[11pt,class=report,crop=false]{standalone}
\usepackage{exo7hilisit}

\newcommand{\pgcd}{\mathop{\mathrm{pgcd}}\nolimits} 
\newcommand{\ppcm}{\mathop{\mathrm{ppcm}}\nolimits}

\begin{document}
    


\entete{Hilisit}{Capacité mathématiques}

\titre{Arithmétique -- Partie 3 : Nombres premiers}

\bigskip
\bigskip


%%%%%%%%%%%%%%%%%%%%%%%%%%%%%%%%%%%%%%%%%%%%%%%%%%%%%%%%%%%%
%\section{Nombres premiers}

\exercice{}
\enonce
Trouver tous les nombres premiers plus petits que $100$.
\finenonce

\indication
Il s'agit d'écarter les entiers qui ne sont pas des nombres premiers car divisibles par $2$ ou par $3$...
\finindication

\correction
Les nombres premiers jusqu'à $100$ sont :
$$2, 3, 5, 7, 11, 13, 17, 19, 23, 29, 31, 37, 41, 43, 47, 53, 59, 61, 67, 71, 73, 79, 83, 89, 97$$

On les obtient simplement par une méthode appelée le \emph{crible d'Ératosthène} :
\begin{itemize}
    \item en excluant d'abord tous les entiers pairs (sauf $2$ bien sûr),
    \item puis tous les entiers divisibles par $3$ (sauf $3$),
    \item on n'a pas besoin d'exclure les multiples de $4$ car ils sont déjà exclus en tant que multiples de $2$,
    \item ensuite on exclut les multiples de $5$ (sauf $5$),
    \item les multiples de $6$ sont déjà exclus (ce sont des multiples de $2$ et de $3$),
    \item il reste à exclure les multiples de $7$,
    \item les multiples de $8$, $9$, $10$ sont déjà exclus,
    \item et c'est terminé car un entier non premier plus petit que $100$ doit avoir un facteur inférieur à $\sqrt{100}=10$.
\end{itemize}
\fincorrection
\finexercice



\exercice{}
\enonce
Calculer la décomposition en facteurs premiers de $a$ puis de $b$, en déduire $\pgcd(a,b)$ et $\ppcm(a,b)$.
\begin{enumerate}
    \item $a=1500$, $b=1470$.
    \item $a=18\,135$, $b=92\,950$.
\end{enumerate}
\finenonce

\indication
Le pgcd et les ppcm s'obtiennent facilement une fois les entiers décomposés en facteurs premiers.
Pour le pgcd prendre, pour chaque facteur premier, l'exposant minimum entre celui de $a$ et celui de $b$, pour le ppcm prendre le maximum.
\finindication

\correction
\sauteligne
\begin{enumerate}
    \item     
    $a=1500 = 2^2 \times 3 \times 5^3$.
    
    Pour obtenir cette décomposition, on remarque que $1500$ est divisible par $2$ donc
    $1500 = 2 \times 750$, puis $750$ est encore divisible par $2$, donc $1500 = 2^2 \times 375$, cette fois $375$ n'est pas divisible par $2$ mais par contre il est divisible par $3$, ainsi $1500 = 2^2 \times 3 \times 125$ et enfin $125=5^3$.
    
    On obtient de même : $\quad b = 2\times 3 \times 5 \times 7^2$.
    
    Pour le pgcd et le ppcm on écrit les entiers avec tous les facteurs présents dans $a$ ou $b$, quitte à mettre des exposants qui valent $0$ :
    $$a=1500 = 2^2 \times 3^1 \times 5^3  \times 7^0$$
    $$b = 1470 = 2^1\times 3^1 \times 5^1 \times 7^2$$
    
    Pour le pgcd on prend, pour chaque facteur premier, l'exposant \emph{minimum} entre celui de $a$ et celui de $b$ :
    $$\pgcd(a,b)=2^1\times 3^1 \times 5^1 \times 7^0 = 30$$
    Pour le ppcm on prend, pour chaque facteur premier, l'exposant \emph{maximum} entre celui de $a$ et celui de $b$ :
    $$\ppcm(a,b)=2^2\times 3^1 \times 5^3 \times 7^2 = 73\,500$$    
    
    \item
    $$a = 18\,135 = 3^2 \times 5 \times 13 \times 31 \qquad b = 92\,950 = 2 \times 5^2 \times 11 \times 13^2$$
    
    $$\pgcd(a,b)=2^0\times 3^0 \times 5^1 \times 11^0 \times 13^1 \times 31^0 = 5 \times 13 =65$$
     
    $$\ppcm(a,b)=2^1\times 3^2 \times 5^2 \times 11^1 \times 13^2 \times 31^1 = 25\,933\,050$$   
   
\end{enumerate}
\fincorrection
\finexercice



\exercice{}
\enonce
Soit $p$ un nombre premier. Montrer que pour tout entier $k$ tel que $1 \leq k \leq p-1$, alors $p$ divise $\binom{p}{k}$.

\textit{On rappelle l'expression du coefficient binomial :}
$$ \binom{n}{k} = \frac{n!}{k!(n-k)!}$$
\finenonce

\indication
Trouver un entier $A$ tel que $\binom{p}{k} = p! \times A$ et utiliser le lemme de Gauss.
\finindication

\correction
Faisons d'abord la remarque suivante : si $a$ et $b$ sont deux entiers, si $p$ est un nombre premier avec $p>a$ et $p>b$ alors bien sûr $p$ ne peut pas diviser $a$, ni $b$ (car $p$ est plus grand que $a$ et $b$) mais en plus $p$ ne peut pas diviser $a \times b$. En effet par le lemme d'Euclide, si $p$ divisait $ab$ alors $p$ diviserait $a$ ou $p$ diviserait $b$.

\bigskip

On sait que $\binom{p}{k} = \frac{p!}{k!(p-k)!} \iff p! = \binom{p}{k} \times k! (p-k)!$. Puisque $p$ divise $p!$, $p$ divise donc $\binom{p}{k} \times k! (p-k)!$. Mais pour $1 \leq k \leq p-1$, tous les facteurs de $k!$ sont strictement inférieurs à $p$ : cela signifie que $p$ ne divise pas $k!$, et donc que $\pgcd(p,k!)=1$. D'après le lemme de Gauss, on a donc : $p$ divise $\binom{p}{k} \times (p-k)!$.\\
Mais il en va de même avec $(p-k)!$ : pour $1 \leq k \leq p-1$, les facteurs de $(p-k)!$ sont tous strictement inférieurs à $p$. Donc $p$ ne divise pas $(p-k)!$, et $\pgcd(p,(p-k)!) = 1$. Une nouvelle application du lemme de Gauss offre donc : 
$$ \boxed{\text{Pour } 1 \le k \le p-1,\qquad   \; p \text{ divise } \binom{p}{k}. \; }$$
\fincorrection
\finexercice


\end{document}


