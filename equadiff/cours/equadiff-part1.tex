\documentclass[11pt,class=report,crop=false]{standalone}
\usepackage{exo7hilisit}

\begin{document}

%%%%%%%%%%%%%%%%%%%%%%%%%%%%%%%%%%%%%%%%%%%%%%%%%%%%%%%%%%%%%%%%%%%%%%
%%%%%%%%%%%%%%%%%%%%%%%%%%%%%%%%%%%%%%%%%%%%%%%%%%%%%%%%%%%%%%%%%%%%%%


\entete{Hilisit}{Capacité mathématiques}

\titre{\'Equations différentielles -- Partie 1 : Primitives} 

\encadre{
	\emph{Savoir.}
	\begin{itemize}[label=$\square$]
		\item Connaître la définition d'une primitive.
		\item Connaître le lien entre deux primitives d'une même fonction.
        \item Connaître les formules des primitives usuelles.
	\end{itemize}
	\emph{Savoir-faire.}
	\begin{itemize}[label=$\square$]
		\item Savoir déterminer une primitive.
	\end{itemize}
}



\bigskip

\subsection*{Primitives}


\begin{itemize}
  \item \textbf{Définition.} Soit $f : I \to \Rr$ une fonction définie sur un intervalle $I$.
On dit qu'une fonction $F$ est une \textbf{primitive} de $f$ sur $I$, si pour tout $x\in I$ :
\mybox{$F'(x)=f(x)$} 

  \item Dans la majorité de nos exemple, les fonctions seront définies sur $\Rr$ tout entier, ainsi si $F'(x)=f(x)$ pour tout $x\in \Rr$ et alors $F$ est une primitive de $f$.  
  %Il est sous-entendu que $F$ doit être dérivable.

  \item Exemples :
\begin{itemize}
	\item $F(x)=\displaystyle\frac{x^3}{3}$ est une primitive de $f(x)=x^2$ (sur $\Rr$) puisque $F'(x) = (\frac{x^3}{3})' = x^2 = f(x)$.
	\item $\ln(x)$ est une primitive de $\frac1x$ sur $]0,+\infty[$.
  %  \item $\ln(x) + 1$ est aussi une primitive de de $\frac1x$ sur $]0,+\infty[$.
\end{itemize}

  \item Trouver une primitive est l'opération inverse du calcul de la dérivée.

  \item \emph{Exercice}. Trouver une primitive de chacune des fonctions suivantes (en précisant l'intervalle $I$ considéré) :
\begin{multicols}{2}
\begin{itemize}
	\item[$\bullet$] $x$
    \item[$\bullet$] $2x - x^2$
    \item[$\bullet$] $\cos(x)$
    \item[$\bullet$] $\sin(x)$
    \item[$\bullet$] $e^{x}$
    \item[$\bullet$] $\dfrac3x - \dfrac{7}{x^2} + 1$

\end{itemize}
\end{multicols}
\end{itemize}


%\subsection*{Calculs d'intégrales à l'aide d'une primitive}
%
%
%\textbf{Théorème.}
%Soit $f$ une fonction définie sur un intervalle $[a,b]$.
%Soit $F$ une primitive de $f$. Alors:
%\mybox{
%$\displaystyle \int_{a}^{b}f(x)\;\dd x=F(b)-F(a)$
%}
%
%\begin{itemize}
%  \item C'est le moyen le plus efficace pour calculer des intégrales ! 
%
%  \item Notation par des crochets. 
%\myboxinline{$\big[F(x)\big]^b_a = F(b)-F(a)$}.
%
%  
%  \item Exemple.
%  \[\int_{1}^{2}x^2\;\dd x=\bigg[\frac{x^3}{3}\bigg]^2_1=\frac{2^3}{3}-\frac{1^3}{3} = \frac{7}{3}.\]
%
%  \item Exemple. \[\int_{2}^{7}\frac1x \;\dd x =\big[ \ln(x) \big]_2^7= \ln(7) - \ln(2) = \ln\left(\tfrac72\right).\]
%\end{itemize}


\subsection*{Toutes les primitives}

\begin{itemize}
   \item Une primitive n'est pas unique ! Soit $f(x)=x^2$, alors $F(x)=\displaystyle\frac{x^3}{3}$ est une primitive. Mais la fonction $G(x) = \displaystyle\frac{x^3}{3} + 2$ est aussi une primitive (dérivez-la pour vérifier). Il y a donc plusieurs primitives. En fait toutes les fonctions $\displaystyle\frac{x^3}{3} + C$, où $C$ est une constante, sont des primitives. Nous généralisons ceci à toutes les fonctions :

 % \item 
  \mybox{\textbf{Proposition.} Si $F(x)$ est une primitive de $f(x)$, alors les autres primitives sont de la forme $F(x)+C$ où $C\in\Rr$ est une constante.}
  
  \item Une conséquence de cette proposition est la suivante : si $F(x)$ et $G(x)$ sont deux primitives d'une même fonction, alors $F$ et $G$ ne diffèrent que d'une constante. Autrement dit, il existe une constante $C$ telle que $F(x) = G(x) + C$.

  \item Exemple. Les primitives de $x^4-3x+5$ sont les fonctions
$\frac{1}{5}x^5 - \frac{3}{2}x^2+5x + C$, où $C\in\Rr$ est une constante.

  \item Exercice. Vérifier que les primitives de la fonction $\frac{1}{\sqrt{x}}$ sont les fonctions $2\sqrt{x} +C$.

\end{itemize}



\subsection*{Primitives usuelles}


\subsubsection*{Primitives des fonctions classiques}

Ici $C$ désigne une constante réelle.
Si l'intervalle n'est pas précisé, c'est $I=\Rr$.

\begin{center}
	\begin{tabular}[t]{|c|c@{\vrule depth 1.2ex height 3ex width 0mm \ }|}
		\hline
		\textbf{Fonction}         & \textbf{Primitives} \\ \hline
		$x^n$          & $\frac{x^{n+1}}{n+1}+C$  \quad ($n \in \Nn$)   \\ \hline
		$\frac{1}{x^n} = x^{-n}$         & $\frac{1}{1-n}\frac1{x^{n-1}}+C=\frac{x^{1-n}}{1-n}+C$  \quad ($n \in \Nn\setminus\{0,1\}$) \quad sur $I=]0,+\infty[$  ou $I=]-\infty,0[$   \\ \hline
		$\frac{1}{x}$  & $\ln(x)+C$ \quad sur $I=]0,+\infty[$              \\ \hline
		$\frac{1}{\sqrt x}$  & $2\sqrt{x}+C$ \quad sur $I=]0,+\infty[$              \\ \hline
%		$x^\alpha$     & $\frac{x^{\alpha+1}}{\alpha+1}+c$  \quad ($\alpha \in \Rr\setminus\{-1\}$)   \\ \hline
%		$\sqrt{x}$    & $\frac23x\sqrt{x}+c=\frac23x^{\frac32}+c$  \quad (c'est $\alpha=\frac12$)   \\ \hline
		$e^x$          & $e^x+C$                        \\ \hline
		$\cos(x)$      & $\sin(x)+C$                \\ \hline
		$\sin(x)$      & $-\cos(x)+C$                     \\ \hline
%		$\tan(x)$      & $-\ln(|\cos(x)|)+c$  \quad ($c\in\Rr$)            \\ \hline
%		$\frac{1}{1+x^2}$      & $\arctan(x)+c$  \quad ($c\in\Rr$)            \\ \hline
	\end{tabular}
\end{center}

Ces formules sont à maîtriser ! Mais ce sont juste les formules des dérivées que vous connaissez déjà.

\subsubsection*{Primitives pour une composition}

Ici $u$ est une fonction dérivable sur un intervalle $I$ ; $C$ désigne une constante réelle.

\begin{center}
	\begin{tabular}[t]{|c|c@{\vrule depth 1.2ex height 3ex width 0mm \ }|}
		\hline
		\textbf{Fonction}         & \textbf{Primitive} \\ \hline
		$u'u^n$         & $\frac{u^{n+1}}{n+1}+C$  \quad ($n \in \Nn$)   \\ \hline
		$u'u^{-n}$         & $\frac{u^{1-n}}{1-n}+C$  \quad ($n \in \Nn\setminus\{0,1\}$)   \quad $u$ ne s'annulant pas sur $I$  \\ \hline
		$\frac{u'}{u}$    & $\ln(u)+C$  où $u(x)>0$ pour tout $x \in I$       \\ \hline
%		$u'u^\alpha$         & $\frac{u^{\alpha+1}}{\alpha+1}+c$  \quad ($\alpha \in \Rr\setminus\{-1\}$)   \\ \hline
		$\frac{u'}{\sqrt{u}}$    & $2\sqrt{u}+C$  \quad où $u(x)>0$ pour tout $x \in I$ \\ \hline
		$u'e^u$         & $e^u+C$                     \\ \hline
		$u'\cos(u)$      & $\sin(u)+C$                      \\ \hline
		$u'\sin(u)$      & $-\cos(u)+C$                        \\ \hline
	%	$u'\tan(u)$      & $-\ln(|\cos(u)|)+c$  \quad ($c\in\Rr$)            \\ \hline
	%	$\frac{u'}{1+u^2}$      & $\arctan(u)+c$  \quad ($c\in\Rr$)            \\ \hline
	\end{tabular}
\end{center}


\begin{itemize}
   \item Exemple.
   Comment calculer une primitive de  $f(x) = x e^{x^2}$ ? 
   Avec $u(x) = x^2$ (et donc $u'(x)=2x$) on a 
   $2x e^{x^2}  = u'(x)e^{u(x)}$ dont une primitive est ainsi $e^{x^2} = e^{u(x)}$. On réécrit alors la fonction dont on recherche une primitive comme $f(x) = \frac 12 \cdot 2x e^{x^2}$ : une primitive est donc $F(x) = \frac 12 e^{x^2}$. 
   Si on veut toutes les primitives, ce sont les fonctions $\frac12 e^{x^2}+C$ où $C$ est une constante.

   \item Exercice. Calculer une primitive de $\cos(x) \big(\sin(x)\big)^2$.

\end{itemize}




\end{document}
